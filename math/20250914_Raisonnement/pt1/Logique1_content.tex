% XeLaTeX can use any Mac OS X font. See the setromanfont command below.
% Input to XeLaTeX is full Unicode, so Unicode characters can be typed directly into the source.

% The next lines tell TeXShop to typeset with xelatex, and to open and save the source with Unicode encoding.

%!TEX TS-program = xelatex
%!TEX encoding = UTF-8 Unicode

\documentclass[a4paper,10pt]{report}
\usepackage[french]{babel}
\usepackage{standalone}
\usepackage{amsmath}
\usepackage{geometry}                % See geometry.pdf to learn the layout options. There are lots.
\geometry{a4paper,hmargin=2cm,vmargin=1.5cm,includeheadfoot}                   % ... or a4paper or a5paper or ... 
%\geometry{landscape}                % Activate for for rotated page geometry
%\usepackage[parfill]{parskip}    % Activate to begin paragraphs with an empty line rather than an indent
\usepackage{graphicx}
\usepackage{amssymb}
\usepackage{tkz-tab}
\usepackage{pgfplots}
\pgfplotsset{compat=1.18}
%\usepackage{xcolor}
%\definecolor{dauphineblue}{RGB}{47,68,134}


%\usepackage{fontspec,xltxtra,xunicode}
%\defaultfontfeatures{Mapping=tex-text}
%\setromanfont[Mapping=tex-text]{Hoefler Text}
%\setsansfont[Scale=MatchLowercase,Mapping=tex-text]{Gill Sans}
%\setmonofont[Scale=MatchLowercase]{Andale Mono}

\title{Logique - Exercices}
\author{Poppy RAVEZ}
%\date{}                                           % Activate to display a given date or no date

\begin{document}
	
	
\subsection*{Exercice 1.1}

\subsubsection*{1.}

Proposition vraie.

Négation : « Xavier n'est pas blond et Xavier n'est pas brun » (« Xavier n'est ni blond ni brun »).

\subsubsection*{2.}

Proposition fausse.

Négation : « Xavier n'est pas roux ou Maria n'est pas blonde »

\subsubsection*{3.}

Proposition vraie.

Négation : « Xavier est brun et Maria n'est pas blonde »


\subsection*{Exercice 1.2}

\subsubsection*{1.}

$NON(x=1 \text{ ou } x=-1) = (x \neq -1 \text{ ET } x \neq 1)$

$= (x \in \mathbb{R}\setminus\{-1;1\}) = (|x|\neq 1)$

\subsubsection*{2.}

$NON(0 \leq x \leq 1) = (x<0 \text{ OU } x>1) $

\subsubsection*{3.}
On peut développer de plusieurs façons :

\begin{equation*}
	\begin{split}
		\text{NON}(x=0 \text{ OU } (x^2 =1 \text{ ET } x\geq 0)) 
		&= \text{NON}(x=0 \text{ OU } ((x=-1\text{ OU } x=1)\text{ ET } x\geq0)\\
		&= \text{NON}(x=0 \text{ OU } x=1)\\
		&= (x\neq 0 \text{ ET } x\neq 1)\\
		\\
		&= (x \neq 0) \text{ ET } (x^2\neq 1 \text{ OU } x<0) \\
		&= (x \neq 0) \text{ ET } ((x \neq -1 \text{ ET } x \neq 1) \text{ OU } x<0))\\
		&= (x \neq 0) \text{ ET } ((x \neq -1 \text{ OU } x <0) \text{ ET }(x \neq 1 \text{ OU } x<0))\\
		&= (x \neq 0) \text{ ET } (x \neq 1 \text{ OU } x<0) \quad\text{ car ($x \neq -1 \text{ OU } x <0)$ est VRAIE}\\
		&= (x \neq 0) \text{ ET } (x \neq 1) \\
		\\
		&= (x \neq 0) \text{ ET } (x^2\neq 1 \text{ OU } x<0) \\
		&= (x\neq 0 \text{ ET } x^2 \neq 1) \text{ OU } (x \neq 0 \text{ ET } x< 0) \\
		&= (x\neq 0 \text{ ET } x \neq -1 \text{ ET } x \neq 1) \text{ OU } (x< 0) \\
		&= (x \neq -1 \text{ OU } x< 0) \text{ ET } (x \neq 0 \text{ OU } x< 0) \text{ ET } (x \neq 1 \text{ OU } x< 0) \\
		&= \text{ VRAI } \text{ ET } x \neq 0 \text{ ET } x \neq 1 \\
		&= x \neq 0 \text{ ET } x \neq 1
	\end{split}
\end{equation*}

car 
\begin{itemize}
	\item $(x \neq 0 \text{ OU } x< 0) \Longleftrightarrow x \neq 0$ \\
	\item $(x \neq 1 \text{ OU } x< 0) \Longleftrightarrow x \neq 1$
\end{itemize}


\subsection*{Exercice 1.3}
Soient les propositions
\begin{itemize}
	\item $P$: « $x<50$ » \\
	\item $Q$: « $x>40$ » \\
\end{itemize}

\subsubsection*{a}
NON($P$) : « $x \geq 50$ »
\subsubsection*{b}
NON($Q$) : « $x \leq 40$ »
\subsubsection*{c}
$P$ OU $Q$ : VRAI
\subsubsection*{d}
NON($P$) OU $Q$ : $Q$
\subsubsection*{e}
NON($P$ OU $Q$) = FAUX
\subsubsection*{f}
NON($P$) ET NON($Q$) : idem cas précédent
\subsubsection*{g}
NON(NON($P$)) = $P$
\subsubsection*{h}
$P$ ET $Q$ = « $x \in ]40 ; 50[$ »


\subsection*{Exercice 1.4}
\subsubsection*{1}
NON(NON($P$) ET NON($Q$)) = $P$ OU $Q$
\subsubsection*{2}
($Q$ ET NON($P)$) OU $P$ = ($Q$ OU $P$) ET (NON($P$) OU $P$) = ($Q$ OU $P$) ET VRAI = $Q$ OU $P$
\subsubsection*{3}

($P$ ET $Q$) OU ($P$ ET NON($Q$)) OU (NON($P$) ET $Q$) \\
= ( ($P$ OU ($P$ ET NON($Q$))) ET ($Q$ OU ($P$ ET NON($Q$))) ) OU (NON($P$) ET $Q$) \\
= ( $P$ ET ($Q$ OU $P$) ) OU (NON($P$) ET $Q$)\\
= $P$ OU (NON($P$) ET $Q$) \\
= $P$ OU $Q$


\subsection*{Exercice 1.5}
\subsubsection*{a}
($P \Longrightarrow$ NON($Q$)) OU (NON($R$ ET $Q$)) \\
= (NON($Q$) OU NON($P$)) OU (NON(VRAI ET VRAI)) \\
= (FAUX OU VRAI) OU FAUX \\
= VRAI

\subsubsection*{b}
(NON($P$) OU NON($Q$)) $\Longrightarrow$ ($P$ OU NON($R$)) \\
= (VRAI OU FAUX) $\Longrightarrow$ (FAUX OU FAUX) \\
= VRAI $\Longrightarrow$ FAUX \\
= FAUX OU NON(VRAI) \\
= FAUX

\subsubsection*{c}
NON(NON($P$) $\Longrightarrow$ NON($Q$)) ET $R$ \\
= NON(VRAI $\Longrightarrow$ FAUX) ET VRAI \\
= NON(FAUX OU NON(VRAI)) \\
= VRAI

\subsubsection*{d}
NON(NON($P$) $\Longrightarrow$ ($Q$ ET NON($R$))) \\
= NON(VRAI $\Longrightarrow$ (VRAI ET FAUX)) \\
= NON(VRAI OU FAUX) \\
= FAUX


\subsection*{Exercice 1.6}
\subsubsection*{1}
$P$ : $x\geq3 \Longrightarrow x+2 \geq 5 $

La contraposée est : $x+2 <5 \Longrightarrow x<3$.

La réciproque est : $x+2 \geq 5 \Longrightarrow x \geq 3$

La proposition est vraie.

\subsubsection*{2}
$Q$ : $n>1 \Longrightarrow n^2 > n $

La contraposée est : $n^2 \leq n \Longrightarrow n \leq 1$.

La réciproque est : $n^2 > n \Longrightarrow n>1$

La proposition est vraie.


\subsection*{Exercice 1.7}

Soit $E$ = « ($P$ ET ($Q$ OU $R$)) $\Longrightarrow$ ($Q$ OU ($P$ ET $R$)) »
\begin{equation*}
	\begin{split}
		E &= [Q \vee (P \wedge R)]\vee NON[P \wedge (Q \vee R)] \\
		  &= Q \vee (P \wedge R) \vee (\bar{P} \vee (\bar{Q} \wedge \bar{R})) \\
		  &= [Q \vee (\bar{Q} \wedge \bar{R})] \vee [\bar{P} \vee (P \wedge R)] \\
		  &= [Q \vee \bar{R}] \vee [\bar{P} \vee R] \\
		  &= R \vee \bar{R} \vee Q \vee \bar{P} \\
		  &= VRAI
	\end{split}
\end{equation*}

\subsubsection*{Négation}
La négation de $E$ est FAUX mais on peut calculer : 

NON( ($Q$ OU ($P$ ET $R$)) OU NON($P$ ET ($Q$ OU $R$)) ) \\
= NON($Q$ OU ($P$ ET $R$)) ET ($P$ ET ($Q$ OU $R$)) \\
= (NON($Q$) ET NON($P$ ET $R$)) ET ($P$ ET ($Q$ OU $R$)) \\
= NON($Q$) ET (NON($P$) OU NON($R$)) ET $P$ ET ($Q$ OU $R$) \\
= [NON($Q$) ET ($Q$ OU $R$)] ET [$P$ ET (NON($P$) OU NON($R$))]  \\
= (NON($Q$) ET $R$) ET ($P$ ET NON($R$)) \\
= $R$ ET NON($R$) ET NON($Q$) ET $P$ \\
= FAUX

\subsubsection*{Contraposée}
La contraposée est : « NON($Q$ OU ($P$ ET $R$)) $\Longrightarrow$ NON($P$ ET ($Q$ OU $R$)) » ;
elle est équivalente à $E$ et donc vraie, on peut néanmoins la calculer :

\begin{equation*}
	\begin{split}
	  C &= NON[P \wedge (Q \vee R)] \vee [Q \vee (P \wedge R)] \\
		&= \bar{P} \vee (\bar{Q} \wedge \bar{R}) \vee Q \vee (P \wedge R) \\
		&= [\bar{P} \vee (P \wedge R)] \vee [Q \vee (\bar{Q} \wedge \bar{R})] \\
		&= [\bar{P} \vee R] \vee [Q \vee \bar{R}] \\
		&= VRAI
	\end{split}
\end{equation*}


\subsubsection*{Réciproque}
La réciproque est : « ($Q$ OU ($P$ ET $R$)) $\Longrightarrow$ ($P$ ET ($Q$ OU $R$)) ».
\begin{equation*}
	\begin{split}
		R &= [P \wedge (Q \vee R)] \vee NON[Q \vee (P \wedge R)] \\
		  &= [P \wedge (Q \vee R)] \vee [\bar{Q} \wedge (\bar{P} \vee \bar{R})] \\
		  &= (P \vee \bar{Q}) \wedge [P \vee (\bar{P} \vee \bar{R})] \wedge [(Q \vee R) \vee \bar{Q}] \wedge [(Q \vee R) \vee (\bar{P} \vee \bar{R})]\\
		  &= (P \vee \bar{Q}) \wedge VRAI \wedge VRAI \wedge VRAI \\
		  &= P \vee \bar{Q}
	\end{split}
\end{equation*}

\begin{tabular}{|c|c|c|c|c|c|c|}
	\hline
	$P$ &  $Q$ & $P \Rightarrow Q$ & NON($P \Rightarrow Q$) & $(P \Rightarrow Q) \Rightarrow P$ & NON$((P \Rightarrow Q) \Rightarrow P)$ & $((P \Rightarrow Q) \Rightarrow P) \Rightarrow P$  \\
	\hline
	V   &  V   & V                 & F                      & V                                 &  F &  V\\
	\hline
	V   &  F   & F                 & V                      & V                                 &  F &  V\\
	\hline
	F   &  V   & V                 & F                      & F                                 &  V &  V\\
	\hline
	F   &  F   & V                 & F                      & F                                 &  V &  V\\
	\hline
\end{tabular}


\subsection*{Exercice 1.8}
\begin{tabular}{|c|c|c|c|c|c|}
	\hline
	(a) & V & A est inclus dans B & (h) & V & $x=4$ ou $x=5$\\
	\hline
	(b) & V & $x=1$ & (i) & V & $x=1$\\
	\hline
	(c) & F & A est inclus dans B & (j) & V & $x=0$ \\
	\hline
	(d) & V & $x=4$ & (k) & F & $\mathbb{N}$ n'a pas de borne supérieure \\
	\hline
	(e) & V & B contient un majorant de C & (l) & F & $D$ pas inclus dans $A$ \\
	\hline
	(f) & V & B contient un majorant de A & (m) & V & 1, 2 ou 3 \\
	\hline
	(g) & V & $x=4$ & (n) & F & Les éléments de A ne sont pas tous égaux \\
	\hline
\end{tabular}


\subsection*{Exercice 1.9}
\subsubsection*{1}
$\forall n \in \mathbb{N}, \exists r \in \mathbb{R}, n = r^2$
\subsubsection*{2}
$\forall n \in \mathbb{N}, \exists r \in \mathbb{R}^{+}, n \leq r$
\subsubsection*{3}
$\exists r \in \mathbb{R}, \forall n \in \mathbb{N}, r \leq n $


\subsection*{Exercice 1.10}
\subsubsection*{1.a}
$\forall x \in \mathbb{R}, f(x)\neq0$
\subsubsection*{1.b}
$\exists (x,v) \in \mathbb{R} \times \mathbb{R}^{*}, f(x) =v$
\subsubsection*{1.c}
$\forall (x,y) \in \mathbb{R}^2, x > y \Longrightarrow f(x) > f(y)$
\subsubsection*{2.a}
« La fonction $f$ est constante sur $\mathbb{R}$. »
\subsubsection*{2.b}
« $f$ est surjective »
\subsubsection*{2.c}
« $f$ présente une borne inférieure qui possède un antécédent »


\subsection*{Exercice 1.11}
\subsubsection*{1.11.1}
« Aucun étudiant n'aime le tennis. »
\subsubsection*{1.11.2}
« Il y a au moins un étudiant qui n'aime pas lire. »
\subsubsection*{1.11.3}
« Il y a au moins une matière dans laquelle aucun étudiant ne travaille régulièrement. »
\subsubsection*{1.11.4}
« Aucun étudiant ne travaille régulièrement dans toutes les matières. »


\subsection*{Exercice 1.12}
$P$ : $((A \wedge B) \Longrightarrow C)$
\subsubsection*{1.12.1}
$P = C \vee \overline{(A \wedge B)} = C \vee (\bar{A} \vee \bar{B}) = \bar{A} \vee \bar{B} \vee C$ = NON($A$) OU NON($B$) OU $C$
\subsubsection*{1.12.2}
$A$ est VRAIE, $B$ est VRAIE et $C$ est FAUSSE, $P$ est donc fausse.


\subsection*{Exercice 1.13}
\subsubsection*{1.13.a}
« Tous les étudiants ont obtenu une note supérieure ou égale à 10 dans toutes les matières. »

FAUX

\subsubsection*{1.13.b}
« Au moins un élève a obtenu une note inférieure à 10 (dans au moins une matière) »

VRAI

\subsubsection*{1.13.c}
« Au moins un élève a eu la moyenne dans toutes les matières. »

FAUX

\subsubsection*{1.13.d}
« Tous les élèves ont la moyenne dans au moins une matières. »

VRAI

\subsubsection*{1.13.e}
« Il existe au moins une matière où aucun élève n'a la moyenne. »

FAUX

\subsubsection*{1.13.f}
« Il y a une matière où tous les élèves ont la moyenne. »

VRAI


\subsection*{Exercice 1.14}
\subsubsection*{1.14.1}
$\forall x \in \mathbb{R}, \exists n \in \mathbb{N}, x \leq n$
\subsubsection*{1.14.2}
$\exists n \in \mathbb{N}, \forall x  \in \mathbb{R}, x \leq 2n$
\subsubsection*{1.14.3}
$\exists(x,y) \in \mathbb{R}^2, (x \geq y) \wedge (x^2 < y^2)$


\subsection*{Exercice 1.15}
\subsubsection*{1.15.1}
$P : \exists x \in \mathbb{R} / \forall y \in \mathbb{R}, x^2 +2xy = 1$

$\bar{P} = \forall x \in \mathbb{R}, \exists y \in \mathbb{R}, x^2+2xy \neq 1$

$P$ est fausse, on démontre en effet que $\bar{P}$ est vraie :
\begin{itemize}
	\item évident pour $x=0$
	\item Pour $x \neq 0$ on pose $y=\frac{2-x^2}{2x}$
\end{itemize}

\subsubsection*{1.15.2}
$P: \forall x \in \mathbb{N}, \exists y \in \mathbb{N}, \exists z \in \mathbb{N}, (x\leq y) \wedge (y=2z)$

$\bar{P}: \exists x \in \mathbb{N} / \forall y \in \mathbb{N}, \forall z \in \mathbb{N}, (x > y) \vee (y \neq 2z)$

$P$ est vraie ($y=2x$ et $z=x$)

\end{document}