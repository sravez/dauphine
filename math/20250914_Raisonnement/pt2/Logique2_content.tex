% XeLaTeX can use any Mac OS X font. See the setromanfont command below.
% Input to XeLaTeX is full Unicode, so Unicode characters can be typed directly into the source.

% The next lines tell TeXShop to typeset with xelatex, and to open and save the source with Unicode encoding.

%!TEX TS-program = xelatex
%!TEX encoding = UTF-8 Unicode

\documentclass[a4paper,10pt]{report}
\usepackage[french]{babel}
\usepackage{standalone}
\usepackage{amsmath}
\usepackage{geometry}                % See geometry.pdf to learn the layout options. There are lots.
\geometry{a4paper,hmargin=2cm,vmargin=1.5cm,includeheadfoot}                   % ... or a4paper or a5paper or ... 
%\geometry{landscape}                % Activate for for rotated page geometry
%\usepackage[parfill]{parskip}    % Activate to begin paragraphs with an empty line rather than an indent
\usepackage{graphicx}
\usepackage{amssymb}
\usepackage{tkz-tab}
\usepackage{pgfplots}
\pgfplotsset{compat=1.18}
%\usepackage{xcolor}
%\definecolor{dauphineblue}{RGB}{47,68,134}


%\usepackage{fontspec,xltxtra,xunicode}
%\defaultfontfeatures{Mapping=tex-text}
%\setromanfont[Mapping=tex-text]{Hoefler Text}
%\setsansfont[Scale=MatchLowercase,Mapping=tex-text]{Gill Sans}
%\setmonofont[Scale=MatchLowercase]{Andale Mono}

\title{Logique - Exercices}
\author{Poppy RAVEZ}
%\date{}                                           % Activate to display a given date or no date

\begin{document}
	
	
\subsection*{Exercice 2.1}

\subsubsection*{2.1.1}
Proposition vraie ($x=0$ et $y=0$)

\subsubsection*{2.1.2}
Proposition fausse.

Soit $x=1$

Alors $\forall y \in \mathbb{N}, x > 0 \geq -y^2$

\subsubsection*{2.1.3}
Proposition vraie.

Soit $y \in \mathbb{N}$.

On pose $x = -y^2 -1$, $x \in \mathbb{Z}$ et donc $x \leq -y^2$.

\subsubsection*{2.1.4}
Proposition vraie.

Soit $a \in \mathbb{N}^*$.

Posons : $b = 2$ et $x= \frac{\ln 3}{a}$

Alors $e^{ax} = 3 > 2 = b$


\subsection*{Exercice 2.2}
\subsubsection*{2.2.1}

Si $\forall x \in \mathbb{R}, f(x)=0$, on a :

$f(0) = 0$ et donc $b=0$

et $f(1) = a = 0$

Donc $a=b=0$

\subsubsection*{2.2.1}

Soit $x$ tel que $f(x) = 0$.

On a donc $b=-ax^2$ et donc $ab = -(ax)^2 \leq 0$


\subsection*{Exercice 2.3}
\subsubsection*{2.3.1}

Soit $\epsilon > 0$.

On pose $\epsilon' = \sqrt{\frac{\epsilon}{5}}$

D'après $(*)$ on sait que :

$\exists x_o \in \mathbb{R}^*_+, 0<f(x_o)<\epsilon'$.

On a donc : $0 < 5f(x_o)^2 < 5\epsilon'^2$

Soit $0 < g(x_o) < \epsilon$

CQFD

\subsubsection*{2.3.2}

Soit $\epsilon > 0$.

D'après $(*)$ on sait que :

$\exists x_1 \in \mathbb{R}^*_+, 0<f(x_1)<\epsilon$.

Et donc $0 < h(\sqrt{x_1}) < \epsilon$

CQFD

\subsubsection*{2.3.3}

Soit $A > 0$, on pose $\varepsilon = \frac{1}{A}$

D'après $(*)$ on sait que :

$\exists x_2 \in \mathbb{R}^*_+, 0<f(x_2)<\varepsilon$.

Et donc $\varphi(x_2^2) = \frac{1}{f(x_2)} > \frac{1}{\varepsilon} = A$

CQFD


\subsection*{Exercice 2.4}
\subsubsection*{2.4.1.a}
La fonction $x^3$ étant strictement croissante :
\begin{alignat*}{2}
	               &\quad & x^3 &=2 \\
	\Longrightarrow&      & x^3 &< 2^3 \\
	\Longrightarrow&      &   x &< 2 
\end{alignat*}

\subsubsection*{2.4.1.b}

Soit $x \in \mathbb{R}$ tel que $x+1$ est le carré d'un entier impair ;
il existe donc $n \in \mathbb{N}$ et :
\begin{alignat*}{2}
	               &\quad & x+1 &=(2n+1)^2 \\
	\Longrightarrow&      & x+1 &= 4n^2 + 4n +1 \\
	\Longrightarrow&      &   x &= 4(n+1) 
\end{alignat*}

\subsubsection*{2.4.2.a}

Soit $a$ et $b$ tels que $b-a > 0$.

On pose $\varepsilon = \frac{(b-a)^2}{2} > 0$

On donc $(b-a)^2 = 2\varepsilon > \varepsilon$

\subsubsection*{2.4.2.b}

Supposons que $b \neq a$ et posons $\varepsilon = |a-b|$.

D'après l'inégalité triangulaire :
$\forall x \in \mathbb{R}, |b-x| + |a-x| \geq |b-a| = \varepsilon$

Ce qui démontre la contraposée.


\subsection*{Exercice 2.5}

\begin{alignat*}{2}
	                   &\quad & xy + 2x + 2y &=-4 \\
	\Longleftrightarrow&      & xy + 2x + 2y +4&=0 \\
	\Longleftrightarrow&      & (x+2)(y+2) & =0 \\
	\Longleftrightarrow&      & (x+2=0) &\vee (y+2=0) \\
	\Longleftrightarrow&      & (x=-2) &\vee (y=-2) \\
\end{alignat*}
CQFD

\subsection*{Exercice 2.6 - Raisonnement par l'absurde}
\subsubsection*{2.6.1}

Supposons que deux entiers $n$ ($n>0$) et $k$ vérifient $n^2 + 1 = k^2$.

$|k|$ est donc strictement positif.

On a alors $(n+k)(n-k) = (n+|k|)(n-|k|)= -1$.

Les deux facteurs sont entiers et doivent donc être de signes différents et avoir une 
valeur absolue égale à 1 ; or $n+|k| >1$ car $n$ et $|k|$ sont strictement positif.

Le postulat de départ est donc faux. CQFD



\subsubsection*{2.6.2}

Soit $n \in \mathbb{N}^*$.

Supposons que $n = k^2$ et $2n = l^2$ avec $(k,l) \in \mathbb{N}^2$.

$k$ et $l$ sont de manière évidente strictement positif

On a alors $k^2 = 2n - n = l^2 -k^2$

Soit $2k^2 = l^2 \Longleftrightarrow  \sqrt{2} = \frac{l}{k} $ ce qui est impossible car $\sqrt{2}$ est irrationnel.

Le postulat de départ est donc faux.

\subsubsection*{2.6.2.b - $\sqrt{c}$ irrationnel si $c$ est premier}

On peut également démontre par l'absurde que $\sqrt{c}$ est irrationnel si c est premier.

Supposons que $\sqrt{c}$ soit rationnel ; il existe donc $p$ et $q$, 2 entiers naturels premiers entre eux tels que
$\sqrt{c}= \frac{p}{q}$.

On a alors $p^2 = cq^2$ donc $c$ divise $p^2$ ; $c$ divise donc $p$ car c'est un nombre premier (si un nombre premier
divise le produit de 2 entiers il divise l'un deux).

Il en résulte que $c^2$ divise $p^2$ et donc également $cq^2$ ; $c$ divise donc $q^2$ et donc $q$.

$c$ est donc un diviseur commun de $p$ et $q$, ce qui contredit l'hypothèse de départ.


\subsection*{Exercice 2.7 - Raisonnement par récurrence}

\subsubsection*{2.7.1}

Soit $P(n)_{n \in \mathbb{N}^*}$ la proposition « $2^n + 3^n \leq 5^n$ »

On a $2^1 + 3^1 = 5 \leq 5^1$ ; $P(1)$ est donc vraie.

Supposons qu'il existe $n \in \mathbb{N}^*$ tel que $P(n)$ soit vraie.

Alors :

$2^{n+1} + 3^{n+1} = 2 \times 2^n + 3 \times 3^n < 3 \times 2^n + 3 \times 3^n = 3(2^n + 3^n) < 5(2^n + 3^n)$

$P(n)$ étant vraie, on a :

$2^{n+1} + 3^{n+1} < 5 \times 5^n = 5^{n+1}$

$P(n+1)$ est donc vraie.

$P$ est donc vraie pour tout entier naturel non nul.


\subsubsection*{2.7.2}

Soit $P(n)$ : « $n! \geq 2^n$ »

Comme $4! = 24 > 16 = 2^4$, $P(4)$ est vraie.

Supposons qu'il existe $n$ entier naturel supérieur ou égal à 4 tel que $P(n)$ soit vraie.

Alors $(n+1)! = (n+1) n! > 2n! \geq 2 \times 2^n = 2^{n+1}$

$P(n+1)$ est donc vraie.

$P$ est donc vraie pour tout entier naturel supérieur ou égal à 4.


\subsubsection*{2.7.3}

Soit $P(n)_{n \in \mathbb{N}^*}$ : « $2^{n-1} \leq n! \geq n^n$ »

On a $1! = 1 \leq 1^1$ et $1! = 1 \geq 2^{1-1}$

$P(1)$ est donc vraie.

Supposons qu'il existe $n \in \mathbb{N}^*$, tel que $P(n)$ soit vraie.

Alors : $(n+1)! = (n+1)n! \leq (n+1)n^n < (n+1)(n+1)^n = (n+1)^{n+1}$

Et : $(n+1)! = (n+1)n! \geq 2n! \geq 2 \times 2^{n-1} = 2^{(n+1)-1}$

$P(n+1)$ est donc vraie.

$P$ est donc vraie pour tout entier naturel non nul.


\subsection*{Exercice 2.8}

S'il existe $(h,\alpha) \in H \times \mathbb{R}, f = \alpha g + h$, on a :

Alors $f(0) = \alpha g(0) + h(0) = \alpha$

Et : $h = f - f(0)g$

S'il existe, le couple est donc unique.

$h$ est une fonction de $\mathbb{R}$ dans $\mathbb{R}$ qui vérifie $h(0) = 0$ et appartient donc bien à $H$

Le couple existe donc et est unique.


\subsection*{Exercice 2.9}
\subsubsection*{2.9.1}
On a :

$f(0) + 0 \times f(1) = 1$, donc $f(0) = 1$

Et donc :

$f(1) + 1 \times f(0) = 2$, donc$f(1) = 1$

\subsubsection*{2.9.2}

En remplaçant $x$ par $1-x$ (possible car $(1-x)$ est une bijection de $\mathbb{R}$ dans $\mathbb{R}$), on a:

$f(1-x) +(1-x)f(x) = 2 -x$

Et donc en composant les deux égalités :

$(1-x(1-x))f(x) + (x-x)f(1-x) = 1+x -x(2-x)$

Soit : $(x^2-x+1)f(x) = x^2 -x +1$

Le polynôme $x^2-x+1$ n'ayant aucune racine, on en conclut que $f$ est la fonction constante égale à 1.


\subsection*{Exercice 2.10}

\subsubsection*{2.10.1.a}

Avec $x=0$ et $y=0$, on a $2f(0) = 4f(0)$ et donc $f(0) = 0$

Avec $x=0$, on a pour tout $y$: $f(y) + f(-y) = 2f(y)$, soit : $f(y)=f(-y)$, $f$ est donc paire.

\subsubsection*{2.10.1.b}

En dérivant deux fois par rapport à $x$ on a :

$\forall (x,y) \in \mathbb{R}^2, f''(x+y) + f''(x-y) = 2f''(x)$

En faisant la même opération par rapport à $y$ on a :

$\forall (x,y) \in \mathbb{R}^2, f''(x+y) + f''(x-y) = 2f''(y)$

Et donc :

$\forall (x,y) \in \mathbb{R}^2, f''(x) = 2f''(y)$

$f''$ est donc constante

\subsubsection*{2.10.2}

$f$ est donc un polynôme du second degré $ax^2 + bx + c$.

$f(0)=0 \Longrightarrow c = 0$

$f$ paire $ \Longrightarrow b = 0$


$f$ est donc de la forme $f(x) = a x^2$ avec $a \in \mathbb{R}$, vérifions que tout ces 
fonctions vérifient la propriété :

$f(x+y) + f(x-y) = a((x+y)^2 +(x-y)^2) = a (2x^2 + 2y^2) = 2(f(x) + f(y))$

L'ensemble des fonctions vérifiant la propriété sont donc celles de la forme $f(x) = a x^2$ avec $a \in \mathbb{R}$.


\end{document}