% XeLaTeX can use any Mac OS X font. See the setromanfont command below.
% Input to XeLaTeX is full Unicode, so Unicode characters can be typed directly into the source.

% The next lines tell TeXShop to typeset with xelatex, and to open and save the source with Unicode encoding.

%!TEX TS-program = xelatex
%!TEX encoding = UTF-8 Unicode

\documentclass[a4paper,10pt]{report}
\usepackage[french]{babel}
\usepackage{standalone}
\usepackage{amsmath}
\usepackage{geometry}                % See geometry.pdf to learn the layout options. There are lots.
\geometry{a4paper,hmargin=2cm,vmargin=1.5cm,includeheadfoot}                   % ... or a4paper or a5paper or ... 
%\geometry{landscape}                % Activate for for rotated page geometry
%\usepackage[parfill]{parskip}    % Activate to begin paragraphs with an empty line rather than an indent
\usepackage{graphicx}
\usepackage{amssymb}
\usepackage{tkz-tab}
\usepackage{pgfplots}
\pgfplotsset{compat=1.18}
%\usepackage{xcolor}
%\definecolor{dauphineblue}{RGB}{47,68,134}


%\usepackage{fontspec,xltxtra,xunicode}
%\defaultfontfeatures{Mapping=tex-text}
%\setromanfont[Mapping=tex-text]{Hoefler Text}
%\setsansfont[Scale=MatchLowercase,Mapping=tex-text]{Gill Sans}
%\setmonofont[Scale=MatchLowercase]{Andale Mono}

\title{Logique - Exercices}
\author{Poppy RAVEZ}
%\date{}                                           % Activate to display a given date or no date

\begin{document}
	
\subsection*{Proposition 3.21 - Distributivité}

On part de la définition des intersection et union et on utilise les règles de distribution
des opérateurs logiques ET ($\wedge$) et OU ($\vee$) dont on remarquera la similarité
sémantique et picturale avec $\cap$ et $\cup$

\subsubsection*{3.21.1 - Distributivité de l'intersection}
\begin{alignat*}{2}
	&                    & x \in A &\cap (B\cup C) \\
	\Longleftrightarrow& & (x \in A) &\wedge ((x \in B) \vee (x \in C)) \\
	\Longleftrightarrow& & ((x \in A) \wedge (x \in B)) &\vee ((x \in A) \wedge (x \in C)) \\
	\Longleftrightarrow& & (x \in A \cap B) &\vee (x \in A\cap C) \\
	\Longleftrightarrow& & x \in (A \cap B) &\cup (A\cap C) \\
\end{alignat*}

\subsubsection*{3.21.1 - Distributivité de l'union}
\begin{alignat*}{2}
	&                    & x \in A &\cup (B\cap C) \\
	\Longleftrightarrow& & (x \in A) &\vee ((x \in B) \wedge (x \in C)) \\
	\Longleftrightarrow& & ((x \in A) \vee (x \in B)) &\wedge ((x \in A) \vee (x \in C)) \\
	\Longleftrightarrow& & (x \in A \cup B) &\wedge (x \in A\cup C) \\
	\Longleftrightarrow& & x \in (A \cup B) &\cap (A\cup C) \\
\end{alignat*}

\subsection*{Exercice 3.1}
\subsubsection*{3.1.1}

$a\mathbb{N} \subset b\mathbb{N} \Longrightarrow a \in b\mathbb{N}$ car $a \in a\mathbb{N}$.

Démontrons la réciproque.

Supposons donc que $a \in b\mathbb{N}$. Par définition de $b\mathbb{N}$, on sait que : $\exists k \in \mathbb{N}, a=kb$

Alors :

$e \in a\mathbb{N} \\
\Longrightarrow \exists k' \in \mathbb{N}, e=k'a \\
\Longrightarrow e=k'kb \\
\Longrightarrow e \in b\mathbb{N}$

Donc : $a\mathbb{N} \subset b\mathbb{N}$.

Ce qui démontre la réciproque et donc l'équivalence


\subsubsection*{3.1.2}

\begin{alignat*}{2}
	                    &\quad &                           a\mathbb{N} &= b\mathbb{N}\\
	\Longleftrightarrow &      &       a\mathbb{N} \subset b\mathbb{N} &\wedge b\mathbb{N} \subset a\mathbb{N} \\
	\Longleftrightarrow &      & \exists (k,k') \in \mathbb{N}^2, a=kb &\wedge b=k'a \\
\end{alignat*}

On a alors $a=kk'a$, donc $k=k'=1$ car $a\neq 0$, donc $a=b$.

La réciproque est évidente.


\subsection*{Exercice 3.2}

L'ensemble des parties de $E$ est donc : $\{ E, \emptyset \}$.

Et $P(P(E)) = \left\lbrace \emptyset, \{\emptyset\}, \{ E\}, \{E, \emptyset \} \right\rbrace $

\subsection*{Exercice 3.3}

Soient $A= \{2, 3, 4 ,5\}$ et $B= \{4,5,6\}$.

Alors :

$A \cap B = \{4,5\}$

$A \cup B = \{2,3,4,5,6\}$

$C_{\mathbb{N}}(A) = \mathbb{N} \setminus \{2, 3, 4 ,5\} = \{0,1,6,7,\ldots\} $ 

$C_{\mathbb{R}}(B) = \mathbb{R} \setminus \{4,5,6\} = ]-\infty;4[ \cup ]4;5[ \cup ]5;6[ \cup]6; +\infty[ $ 


\subsection*{Exercice 3.4}

Soient $A = \{ x \in \mathbb{R} / x^2-3x+1 >0\}$ et $B = \{x \in \mathbb{R} /x >0\}$


Les racines de $x^2 -3x +1$ sont $r_1 = \frac{3-\sqrt{5}}{2}$ et $r_2 = \frac{3+\sqrt{5}}{2}$


Alors :

$A = ]-\infty; r_1[ \cup ]r_2 ; +\infty[$

$B = \mathbb{R}^*_+$

$A^c = \{ x \in \mathbb{R} / x^2-3x+1 \leq 0\} = [r_1 ; r_2]$

$B^c = \mathbb{R}_{-}$

$A \cap B = ]0; r_1[ \cup ]r_2 ; +\infty[$

$A \cup B = \mathbb{R}$

$A \setminus B = ]-\infty; 0]$

$B \setminus A = [r_1; r_2]$

\subsection*{Exercice 3.5}

\subsubsection*{1}

Supposons que $A \subset B$

$A \subset B \Longrightarrow (A \cup B) \subset (B \cup B) \Longrightarrow (A \cup B) \subset B$

Or $B \subset (A \cup B)$, donc $B = (A \cup B)$.

On a donc $(a) \Longrightarrow (b)$ :

$A \subset B \Longrightarrow B = (A \cup B)$

\subsubsection*{2}

Supposons que $B = (A \cup B)$

Alors :

$A \cap B = A \cap (A \cup B) = (A \cap A) \cup (A \cap B) = A \cup (A \cap B) = A$ car par définition $(A \cap B) \subset A$.

On a donc $(b) \Longrightarrow (c)$ :

$B = (A \cup B) \Longrightarrow A \cap B = A$

\subsubsection*{3}

Supposons que $A \cap B = A$

Tout élément de $A$ appartient donc à l'intersection de $A$ et $B$, donc à $B$ soit $A \subset B$

On a donc $(c) \Longrightarrow (a)$ :

$A \cap B = A \Longrightarrow A \subset B$

\subsubsection*{Conclusion}

On a donc $(a) \Longrightarrow (b) \Longrightarrow (c) \Longrightarrow (a)$, ce qui démontre l'équivalence
des trois propositions.


\subsection*{Exercice 3.6}
\subsubsection*{1}
Il est nécessaire que $E$ soit non vide.

Si $A=E$, $B=E$, et $C=\emptyset$, on a $A \cup B = A \cup C = E$ 

\subsubsection*{2}
Il est nécessaire que $E$ soit non vide.

Si $A=\emptyset$, $B=E$, et $C=\emptyset$, on a $A \cap B = A \cap C = \emptyset$ 

\subsubsection*{3}

Supposons que $(A \cup B = A \cup C) \wedge (A \cap B = A \cap C)$.

Soit $x \in B$.

Si $x \in A \cap B$, $x \in A \cap C$ d'après les égalités ci-dessus, et donc $x \in C$.

Si $x \notin A \cap B$, $x \notin A$ mais $x \in A \cup C$ d'après les inégalités ci-dessus (car appartient à $A \cup B$), et donc $x \in C$.

Donc $x \in C$, ce qui démontre l'implication.

\subsection*{Exercice 3.7}

\subsubsection*{1}
\begin{alignat*}{2}
	                    &\quad &               x \in (A \setminus B)&\setminus C  \\
	\Longleftrightarrow &      &       (x \in A \setminus B) &\wedge (x \notin C) \\
	\Longleftrightarrow &      &  (x \in A) \wedge (x \notin B) &\wedge (x \notin C)\\
	\Longleftrightarrow &      &  (x \in A) \wedge [NON(x \in B) &\wedge NON(x \in C)]\\
	\Longleftrightarrow &      &  (x \in A) \wedge NON[(x \in B) &\vee (x \in C)]\\
	\Longleftrightarrow &      &  (x \in A) \wedge NON(x \in B&\cup C)\\
	\Longleftrightarrow &      &  (x \in A) \wedge (x \notin B&\cup C)\\
    \Longleftrightarrow &      &  x \in A \setminus(B&\cup C)
\end{alignat*}

CQFD

\subsubsection*{2}
\begin{alignat*}{2}
	                    &\quad &               x \in A &\cap (A^c \cup B)  \\
	\Longleftrightarrow &      &  (x \in A) &\wedge (x \in (A^c \cup B))\\
	\Longleftrightarrow &      & (x \in A) &\wedge ( (x \in A^c) \vee (x \in B))  \\
	\Longleftrightarrow &      &  (x \in A) &\wedge ( NON(x \in A) \vee (x \in B))\\
	\Longleftrightarrow &      &  ((x \in A) \wedge NON(x \in A)) &\vee ((x \in A) \wedge (x \in B))\\
	\Longleftrightarrow &      &  FAUX &\vee (x \in A \cap B)\\
	\Longleftrightarrow &      &  x \in A &\cap B
\end{alignat*}

CQFD


\subsubsection*{3}

\begin{alignat*}{2}
	                    &\quad &  C_E(B) &\subset C_E(A)  \\
	\Longleftrightarrow &      & \forall x \in E, &(x \in C_E(B)) \Longrightarrow (x \in C_E(A)) \\
	\Longleftrightarrow &      & \forall x \in E, &(x \in C_E(A)) \vee NON(x \in C_E(B))\\
	\Longleftrightarrow &      & \forall x \in E, &NON(x \in A) \vee (x \in B)\\
	\Longleftrightarrow &      & \forall x \in E, &(x \in A) \Longrightarrow (x \in B)\\
	\Longleftrightarrow &      & A &\subset B
\end{alignat*}

CQFD

\subsection*{Exercice 3.8}

\begin{alignat*}{2}
	                    &\quad &               x \in E &\setminus (F \cap G)  \\
	\Longleftrightarrow &      & (x \in E) &\wedge NON(x \in F\cap G)\\
	\Longleftrightarrow &      & (x \in E) &\wedge NON( (x \in F) \wedge (x \in G) )\\
	\Longleftrightarrow &      & (x \in E) &\wedge ( NON(x \in F) \vee NON(x \in G) )\\
	\Longleftrightarrow &      & [ (x \in E) \wedge NON(x \in F) ] &\vee [ (x \in E) \wedge NON(x \in G) ]\\
	\Longleftrightarrow &      & (x \in E\setminus F) &\vee (x \in E\setminus G)\\
	\Longleftrightarrow &      & x \in (E\setminus F) &\cup ( E\setminus G)
\end{alignat*}

CQFD

\subsection*{Exercice 3.9}

On remarque que les propriétés sont inchangées si on effectue les permutations ($A \leftrightarrow B$) et ($C \leftrightarrow D$).
Il suffit donc de démontrer la première égalité, la seconde en découllant par permutation.

Soit $x \in C$.

D'après (iii), $x \notin D$ et donc, d'après (ii), $x \notin B$.

D'après (iv), $x \in A \cup B$ (car dans $C \cup D$) et donc $x \in A$ (car n'appartient pas à $B$.)

Il en résulte que $C \subset A$, ce qui implique d'après (i) que $A = C$


\subsection*{Exercice 3.10}

$E = \mathbb{Z} \times 2\mathbb{N} \times [-5 ; 5]$ si on entend entier naturel par nombre pair.


\subsection*{Exercice 3.11}

\begin{alignat*}{2}
	                    &\quad &               (x,y) &\in (A \times B) \cap (C \times D)  \\
	\Longleftrightarrow &      &     [(x,y) \in (A \times B)] &\wedge [(x,y) \in (C \times D)]\\
	\Longleftrightarrow &      & [(x \in A) \wedge (y \in B)] &\wedge [(x \in C) \wedge (y \in D)]\\
	\Longleftrightarrow &      & (x \in A) \wedge (x \in C)   &\wedge (y \in B) \wedge (y \in D)\\
	\Longleftrightarrow &      &           (x \in A \cap C)   &\wedge (y \in B \cap D)\\
	\Longleftrightarrow &      &                        (x,y) &\in (A \cap C) \times (B \cap D)
\end{alignat*}

CQFD

\subsection*{Exercice 3.12}

Soit $A_k = [0;k]$ pour $k \in \mathbb{N}$

\subsubsection*{3.12.1}
$A_0 = \{0\}$ par définition

\subsubsection*{3.12.2}
$A_1 = [0 ; 1]$ par définition

\subsubsection*{3.12.3}
$A_1 \setminus A_2 = \emptyset$ car $A_1 \subset A_2$

\subsubsection*{3.12.4}
$A_2 \setminus A_1 = [0;2]\setminus [0;1] = ]1 ; 2]$

\subsubsection*{3.12.5}

Par définition l'intersection est incluse dans $A_0$ (comme dans tous les intervalles) ; $A_0$ étant inclus
dans tous les intervalles, il est inclus dans l'interdiction. Il en résulte que l'intersection est $A_0$.

$\bigcap_{k \in \mathbb{N}} A_k = A_0 = \{0\}$

\subsubsection*{3.12.6}

L'union est incluse dans $\mathbb{R}_+$ (car tous les intervalles sont inclus dans $\mathbb{R}_+$)

Tout réel positif $x$ est compris dans $A_{\lfloor x \rfloor + 1}$ (où $\lfloor x \rfloor$ est la partie entière de $x$) et
donc dans l'union.

Il en résulte que $\bigcup_{k \in \mathbb{N}} A_k = \mathbb{R}_+$ 


\subsection*{Exercice 3.13}

Soit $A_k = [k;k+10]$ et $B_k = [-1 ; k]$ pour $k \in \mathbb{N}$

\subsubsection*{E1}

\begin{alignat*}{2}
 	                    &\quad & x &\in \bigcap_{k=i}{j} A_k  \\
	\Longleftrightarrow &      & \forall k \in \{i,\ldots,j\},& x\geq k \wedge x \leq k+10  \\
	\Longleftrightarrow &      &  x\geq j &\wedge x \leq i+10 \\
	\Longleftrightarrow &      & x &\in [j ; i+10]           
\end{alignat*}

$E_1 = [3;13]\cap[4;14]\cap[5;15]\cap \ldots \cap [10 ; 20] = [10 ; 13]$

\subsubsection*{E2}

$E_2 = \bigcap_{k \in \mathbb{N}} A_k = \emptyset$ car par exemple $A_0 \cap A_{11} = \emptyset$

\subsubsection*{E3}

On démontre facilement par récurrence que $\bigcup_{k=i}^{j} A_k = [i ; j+10]$

$E_3 = \bigcup_{k=3}^{10} A_k= [3 ; 20]$

\subsubsection*{E4}

L'union des intervalles est inclus dans $\mathbb{R}_+$.

Tout réel positif $x$ est compris dans $A_{\lfloor x \rfloor}$ et donc dans l'union des intervalles.

$E_4 = \bigcup_{k \in \mathbb{N}} A_k = \mathbb{R}_+$

\subsubsection*{F1}

$F_1 = \bigcap_{k=3}^{10} B_k= B_3 = [-1 ; 3]$ car $B_k \subset B_{k+1}$

\subsubsection*{F2}

L'intersection est par définition un sous-ensemble de $B_0$ ; $B_0$ étant un sous-ensemble de tout $B_k$, il appartient 
à l'intersection.

$F_2 = \bigcap_{k \in \mathbb{N}} B_k= B_0 = [-1 ; 0]$

\subsubsection*{F3}


$F_3 = \bigcup_{k=3}^{10} B_k = B_{10} = [-1 ; 10]$

\subsubsection*{F4}


$F_4 = \bigcup_{k \in \mathbb{N}} B_k = [-1 ; +\infty[$


\subsection*{Exercice 3.14}

\subsubsection*{1}

$x \in I_p \Longrightarrow x \leq \frac{(p+1)\alpha}{m}$

$x \in I_{p+1} \Longrightarrow x > \frac{(p+1)\alpha}{m}$

Les intervalles sont donc disjoints, $I_p \cap I_{p+1} = \emptyset$

\subsubsection*{2}
$\bigcup_{p=0}^{m-1} = ]0 ; \alpha]$

\end{document}