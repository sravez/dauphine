% XeLaTeX can use any Mac OS X font. See the setromanfont command below.
% Input to XeLaTeX is full Unicode, so Unicode characters can be typed directly into the source.

% The next lines tell TeXShop to typeset with xelatex, and to open and save the source with Unicode encoding.

%!TEX TS-program = xelatex
%!TEX encoding = UTF-8 Unicode

\documentclass[a4paper,10pt]{report}
\usepackage[french]{babel}
\usepackage{standalone}
\usepackage{amsmath}
\usepackage{geometry}                % See geometry.pdf to learn the layout options. There are lots.
\geometry{a4paper,hmargin=2cm,vmargin=1.5cm,includeheadfoot}                   % ... or a4paper or a5paper or ... 
%\geometry{landscape}                % Activate for for rotated page geometry
%\usepackage[parfill]{parskip}    % Activate to begin paragraphs with an empty line rather than an indent
\usepackage{graphicx}
\usepackage{amssymb}
\usepackage{tkz-tab}
\usepackage{pgfplots}
\pgfplotsset{compat=1.18}
%\usepackage{xcolor}
%\definecolor{dauphineblue}{RGB}{47,68,134}


%\usepackage{fontspec,xltxtra,xunicode}
%\defaultfontfeatures{Mapping=tex-text}
%\setromanfont[Mapping=tex-text]{Hoefler Text}
%\setsansfont[Scale=MatchLowercase,Mapping=tex-text]{Gill Sans}
%\setmonofont[Scale=MatchLowercase]{Andale Mono}

\title{Math - Exercices}
\author{Poppy RAVEZ}
%\date{}                                           % Activate to display a given date or no date

\begin{document}

\chapter{Cours}

\subsection*{Application 3.14}
\begin{displaymath}
	\int_1^x \ln y dy = [y\ln y]_1^x -\int_1^x y.\frac{1}{y}dy = x \ln x - [y]_1^x = x(\ln x -1) + 1
\end{displaymath}

\subsection*{Exercice 3.1}
\begin{equation*}
	\begin{split}
	\int_0^3 f(x)dx &= \int_0^1 f(x)dx + \int_1^2 f(x)dx +\int_2^3 f(x)dx\\
	                &= \int_0^1 x dx + \int_1^2 1 dx + \int_2^3 (x-1)^2dx \\
	                &= \left[ \frac{x^2}{2} \right]_0^1 +  \left[ x \right]_1^2 + \left[ \frac{(x-1)^3}{3} \right]_2^3 \\
	                &= \frac{1}{2} + 2 - 1 + \frac{8-1}{3} \\
	                &= \frac{3+6+14}{6} \\
	                &= \frac{23}{6}
	\end{split}
\end{equation*}

\subsection*{Exercice 3.2 - Intégrales et fonctions composées}

\subsubsection*{3.2.a}
\begin{equation*}
	\begin{split}
		\int_2^3 \frac{x}{x^2-1}dx &= \left[ \frac{1}{2}\ln(x^2-1) \right]_2^3\\
			                       &= \frac{1}{2} (\ln 8 - \ln 3) \\
			                       &= \frac{1}{2} \ln \frac{8}{3} \\
			                       &= \ln \sqrt{\frac{8}{3}} \\
			                       &= \ln 2\sqrt{\frac{2}{3}}
	\end{split}
\end{equation*}

\subsubsection*{3.2.b}
\begin{equation*}
	\begin{split}
		\int_2^3 \frac{x}{(x^2+1)^3}dx &= \left[ \frac{-1}{4}\frac{1}{(x^2+1)^2} \right]_2^3\\
		                               &= \frac{-1}{4} \left( \frac{1}{100} - \frac{1}{25} \right) \\
		                               &= \frac{3}{400}
	\end{split}
\end{equation*}

\subsubsection*{3.2.c}
\begin{equation*}
	\begin{split}
		\int_1^2 \frac{\ln x}{x}dx &= \left[\frac{1}{2}\ln^2 x \right]_1^2\\
		                           &= \frac{\ln^2 2}{2}
	\end{split}
\end{equation*}


\subsection*{Exercice 3.3 - Intégration par parties}
\subsubsection*{3.3.a}
\begin{equation*}
	\begin{split}
		\int_0^1 xe^{-x}dx &= \left[ -xe^{-x} \right]_0^1 - \int_0^1 -e^{-x}dx \\
		                   &= -e^{-1} - \left[ e^{-x} \right]_0^1 \\
		                   &= -e^{-1} - e^{-1} + 1 \\
		                   &= 1 - \frac{2}{e} \\
		                   &= \frac{e-2}{e}
	\end{split}
\end{equation*}

\subsubsection*{3.3.b}
\begin{equation*}
	\begin{split}
		\int_0^1 x^2e^xdx &= \left[ x^2 e^x \right]_0^1 - \int_0^1 2x e^x dx \\
		                  &= e - 2 \left[ x e^x \right]_0^1 + 2\int_0^1 e^x dx \\
		                  &= e -2e + 2 \left[ e^x \right]_0^1 \\
		                  &= -e + 2(e -1) \\
		                  & = e - 2
	\end{split}
\end{equation*}

\subsubsection*{3.3.c}
\begin{equation*}
	\begin{split}
		\int_1^x y \ln y dy &= \left[ \frac{y^2}{2} \ln y \right]_1^x - \int_1^x \frac{y}{2} dy \\
		                    &= \frac{x^2}{2} \ln x - \left[ \frac{y^2}{4} \ln y \right]_1^x \\
		                    &= \frac{x^2}{2} \ln x - \frac{x^2 -1}{4}
	\end{split}
\end{equation*}

\subsubsection*{3.3.d}
\begin{equation*}
	\begin{split}
		\int_1^3 \ln^2(x) dx &= \left[ x \ln^2(x) \right]_1^3 - \int_1^3 2\ln(x) dx \\
		                     &= 3 \ln^2(3) - \left[ 2x \ln(x) \right]_1^3 +  \int_1^3 2 dx \\
		                     &= 3 \ln^2(3) -6 \ln(3) + \left[ 2x \right]_1^3 \\
		                     &= 3 \ln^2(3) -6 \ln(3) + 4
	\end{split}
\end{equation*}

\subsection*{Exercice 3.4 - Décomposition en éléments simples}

\subsubsection*{3.4.1}
Soit 
\begin{equation*}
	\begin{split}
		f(x) &= \frac{1}{2x^2 + 3x -2} \\
		       &= \frac{1}{(x+2)(2x-1)} 
	\end{split}
\end{equation*}


\begin{alignat*}{2}
 &                     &                f(x) &= \frac{a}{x+2} +  \frac{b}{2x-1}\\
 & \Longleftrightarrow & \frac{1}{(x+2)(2x-1)} &= \frac{(2a+b)x + 2b -a}{(x+2)(2x-1)} \\
 & \Longleftrightarrow & (2a+b =0)             &\vee (2b-a = 1) \\
 & \Longleftrightarrow & a=\frac{-1}{5}        &\vee b = \frac{2}{5}
\end{alignat*}




Et donc :

\begin{displaymath}
	f(x) = \frac{-1}{5}\frac{1}{x+2} +  \frac{2}{5}\frac{1}{2x-1}
\end{displaymath}

\subsubsection*{3.4.2}

Comme $(x+2)$ et $(2x-1)$ sont strictement positifs sur $]\frac{1}{2};+\infty[$, une primitive de la fonction sur cet
intervalle est :

\begin{displaymath}
	P = \frac{-1}{5}\ln(x+2)+  \frac{1}{5}\ln(2x-1) = \frac{1}{5}\ln \frac{2x-1}{x+2}
\end{displaymath}

\subsubsection*{3.4.3}

Soit :

\begin{equation*}
	f_1(x) =\frac{3}{x^2 + 2x -3} = \frac{3}{(x+3)(x-1)}
\end{equation*}

Cherchons à décomposer en éléments simples :

\begin{alignat*}{2}
	&                     &                f_1(x) &= \frac{a}{x+3} +  \frac{b}{x-1}\\
	& \Longleftrightarrow & \frac{3}{(x+3)(x-1)}  &=  \frac{(a+b)x + 3b -a}{(x+3)(x-1)} \\
	& \Longleftrightarrow & (a+b=0)               &\vee (3b-a = 3) \\
	& \Longleftrightarrow & a=\frac{-3}{4}        &\vee b = \frac{3}{4}
\end{alignat*}

Donc :

\begin{displaymath}
	f_1 = \frac{3}{4} \left( \frac{1}{x-1} - \frac{1}{x+3} \right)
\end{displaymath}

$f_1$ possède donc comme primitive en tenant compte du signe de $x-1$ et $x+3$ sur les différents intervalles :

\begin{displaymath}
	P_1 = 
	\begin{cases}
		\frac{3}{4} \ln \frac{x+3}{x-1}      & \text{ sur } ]-\infty ; -3]\\
		\frac{-3}{4} \ln(-3x^2-2x+3)         & \text{ sur } ]-3 ; 1[\\
		\frac{3}{4} \ln \frac{x-1}{x+3}      & \text{ sur } ]1 ; +\infty[
	\end{cases}
\end{displaymath}


Soit :
\begin{equation*}
		f_2(x) =\frac{2x}{x^2 + 2x -3} = \frac{2x}{(x+3)(x-1)}
\end{equation*}

Cherchons à décomposer en éléments simples :

\begin{alignat*}{2}
	&                     &                f_2(x)  &= \frac{a}{x+3} +  \frac{b}{x-1}\\
	& \Longleftrightarrow & \frac{2x}{(x+3)(x-1)}  &=  \frac{(a+b)x + 3b -a}{(x+3)(x-1)} \\
	& \Longleftrightarrow & (a+b=2)               &\vee (3b-a = 0) \\
	& \Longleftrightarrow & a=\frac{3}{2}        &\vee b = \frac{1}{2}
\end{alignat*}

Donc :

\begin{displaymath}
	f_2 = \frac{1}{2} \left( \frac{1}{x-1} + \frac{3}{x+3} \right)
\end{displaymath}

Une primitive de $f_2$ est donc :
\begin{displaymath}
	P_2 = 
	\begin{cases}
		-\frac{1}{2}\ln(1-x) -\frac{3}{2}\ln(-3-x) = -\frac{1}{2} \ln((x-1)(x+3)^3)      & \text{ sur } ]-\infty ; -3]\\
		-\frac{1}{2}\ln(1-x) +\frac{3}{2}\ln(3+x)  = \frac{1}{2} \ln \frac{(x+3)^3}{1-x} & \text{ sur } ]-3 ; 1[\\
		\frac{1}{2}\ln(x-1) +\frac{3}{2}\ln(3+x)   = \frac{1}{2} \ln((x-1)(x+3)^3)       & \text{ sur } ]1 ; +\infty[
	\end{cases}
\end{displaymath}

\end{document}