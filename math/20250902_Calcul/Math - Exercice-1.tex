% XeLaTeX can use any Mac OS X font. See the setromanfont command below.
% Input to XeLaTeX is full Unicode, so Unicode characters can be typed directly into the source.

% The next lines tell TeXShop to typeset with xelatex, and to open and save the source with Unicode encoding.

%!TEX TS-program = xelatex
%!TEX encoding = UTF-8 Unicode

\documentclass{report}
\usepackage[french]{babel}
\usepackage{amsmath}
\usepackage{geometry}                % See geometry.pdf to learn the layout options. There are lots.
\geometry{a4paper,hmargin=2cm,vmargin=1.5cm,includeheadfoot}                   % ... or a4paper or a5paper or ... 
%\geometry{landscape}                % Activate for for rotated page geometry
%\usepackage[parfill]{parskip}    % Activate to begin paragraphs with an empty line rather than an indent
\usepackage{graphicx}
\usepackage{amssymb}



%\usepackage{fontspec,xltxtra,xunicode}
%\defaultfontfeatures{Mapping=tex-text}
%\setromanfont[Mapping=tex-text]{Hoefler Text}
%\setsansfont[Scale=MatchLowercase,Mapping=tex-text]{Gill Sans}
%\setmonofont[Scale=MatchLowercase]{Andale Mono}

\title{Math - Exercices}
\author{Poppy RAVEZ}
%\date{}                                           % Activate to display a given date or no date

\begin{document}


\chapter{Calcul algébrique}

\section*{Application 1.2}

\subsubsection*{1. Développement}
\begin{equation*}
\begin{split}
	(a+b+c)^{2} &= (a+b+c)(a+b+c) \\
	            &= a^{2}+ab+ac+ba+b^{2}+bc+ca+cb+c^{2}\\
	            &= a^{2}+b^{2}+c^{2} + 2(ab+ac+bc)
\end{split}	
\end{equation*}
ou
\begin{equation*}
\begin{split}
	(a+b+c)^{2} &= ((a+b)+c)^2 \\
	&= (a+b)^{2}+c^{2}+2(a+b)c\\
	&= a^{2}+b^{2}+ 2ab +c^{2} +2ac+2bc) \\
	&= a^{2}+b^{2}+c^{2} +2(ab+ac+bc)
\end{split}	
\end{equation*}

\subsubsection*{2. Calcul}

\begin{equation*}
	\begin{split}
		10\,002 \times 99\,998 &= (10^4 +2)(10^4 - 2) \\
		                       &= 10^8 - 4 \\
		                       &= 99\,999\,996
	\end{split}
\end{equation*}

\begin{equation*}
	\begin{split}
		100\,001^2 &= (10^5 + 1)^2 \\
		           &= 10^{10} + 2 \times 10^5 + 1 \\
		           &= 10\,000\,200\,001
	\end{split}
\end{equation*}



\subsubsection*{3. Factorisation}

\begin{equation*}
	\begin{split}
		(2x-5)^2 - (2x - 9)^2 &= (2x-5+2x-9)(2x-5-2x+9) \\
		&= 4(4x-14) \\
		&= 8(2x-7)
	\end{split}
\end{equation*}



\section*{Application 1.8}

\subsubsection*{1.}

\begin{displaymath}
	\binom{n}{0}= \frac{n!}{0!(n-0)!}= \frac{n!}{1 \times n!}=1
\end{displaymath}

\subsubsection*{2.}
\begin{displaymath}
	\binom{n}{n}= \frac{n!}{(n-n)!n!}= \frac{n!}{0!n!}=1
\end{displaymath}


\subsubsection*{3.}
\begin{displaymath}
	\binom{n}{k}= \frac{n!}{(n-k)!k!}= \frac{n!}{(n-k)!(n-(n-k))!}=\binom{n}{n-k}
\end{displaymath}



\section*{Proposition 1.9 - Raisonnement de dénombrement}

Soit $E$ un ensemble de $n$ éléments et $e$ un de ces éléments.

Le nombre de combinaisons de $k$ éléments contenant $e$ est $N_e = \binom{n-1}{k-1}$ : chaque combinaison est
constituée de $e$ et de $k-1$ éléments dans les $n-1$ restants.

Le nombre de combinaisons de $k$ éléments ne contenant pas $e$ est $N_{\bar{e}} = \binom{n-1}{k}$ : il faut choisir $k$ éléments dans les $n-1$ restants.

Donc :
\begin{displaymath}
\binom{n}{k}= N_e + N_{\bar{e}} =  \binom{n-1}{k-1}+ \binom{n-1}{k}
\end{displaymath}


\section*{Application 1.10}

Soit la proposition $P(n)_{n \in \mathbb{N}^*}$ : « $\forall  k \in \left\lbrace 1,\ldots, n\right\rbrace, \binom{n}{k} \in \mathbb{N}^*$ »

$P(1)$ est vraie

Supposons que  $P(N)$ avec $N \in \mathbb{N}^*$ est vrai ; on a :

\begin{equation*}
	\begin{cases}
		\binom{N+1}{0} = 1 \implies \binom{N+1}{0} \in \mathbb{N}^* \\
		\forall k \in \left\lbrace 1, ... , N\right\rbrace , \binom{N+1}{k} = \binom{N}{k} + \binom{N}{k-1} \in \mathbb{N}^* \text{ car somme de 2 entiers naturels non nuls}\\
		\binom{N+1}{N+1} = 1  \implies \binom{N+1}{N+1} \in \mathbb{N}^*
	\end{cases}
\end{equation*}

$P(N+1)$ est donc vraie


\section*{Proposition 1.16}

Si la raison est différente de $1$, la somme d'une suite géométrique est : « le suivant moins le premier sur la raison moins $1$».

\section*{Application 1.22}

\subsubsection*{1. Somme de 7 éléments consécutifs d'une suite arithmétique}
\begin{displaymath}
	2 + 3 + 4 + 5 + 6 + 7 + 8 = \frac{(8+2)\times 7}{2} = 35
\end{displaymath}

\subsubsection*{2. Somme de termes consécutifs d'une suite géométrique}
\begin{displaymath}
\sum_{k = 0}^{4} 3^k = \frac{3^5-1}{3-1}=\frac{243-1}{2}=121
\end{displaymath}


\subsubsection*{3. Sommes de coefficients binomiaux} 
\begin{displaymath}
	\sum_{k = 0}^{n} \binom{n}{k} = \sum_{k = 0}^{n} \binom{n}{k} 1^k . 1^{n-k} = (1+1)^n = 2^n
\end{displaymath}
\begin{displaymath}
	\sum_{k = 0}^{n} (-1)^k \binom{n}{k} = \sum_{k = 0}^{n} \binom{n}{k} (-1)^k . 1^{n-k} = (1-1)^n = 0
\end{displaymath}


\section*{Application 1.27 - Inégalités}

\subsubsection*{1. Changement de sens}
\begin{alignat*}{3}
	        &         &          x &\leq y \\
	\implies& \qquad  &   x+(-x-y) &\leq y+(-x-y) &\qquad& \text{d'après (b)}\\
	\implies&         &         -y &\leq -x
\end{alignat*}


\subsubsection*{2. Addition d'inégalités}

L'addition d'un réel à une inégalité (b) permet de dire :
\begin{displaymath}
	\begin{split}
x \leq y & \implies x  + z\leq y + z \\
z \leq t & \implies z + y \leq t + y
	\end{split}
\end{displaymath}

Donc d'après la transitivité (c) : $ x  + z \leq y + t$ 

\subsubsection*{3. Soustraction d'inégalités}
 
Changement de sens : $z \leq t \implies -t \leq -z$ (cf. ci-dessous)

Addition d'inégalités : $x - t \leq y - z$ 

\subsubsection*{4. Produit d'un réel négatif avec une inégalités}

\begin{alignat*}{3}
	        &         &     z &< 0 \\
	\implies& \qquad  &    -z &> 0        &\qquad& \text{Changement de sens}\\
	\implies&         & x(-z) &\leq y(-z) &      & \text{d'après (d)}\\
	\implies&         &    xz &\geq yz    &      & \text{Changement de sens}
\end{alignat*}

\subsubsection*{5. Produit d'inégalités à termes positifs}

\begin{alignat*}{3}
	        &\qquad &  x \leq y  &\wedge z\leq t\\
	\implies&       & xz \leq yz &\wedge zy \leq ty &\qquad &\text{d'après (d) car $z$ et $y$ sont positifs} \\
	\implies&       &         xz &\leq yt           &       &\text{d'après (c), transitivité}
\end{alignat*}

\section*{Application 1.30}
\subsubsection*{1.}

\[(a+b)^2 - 4ab = a^2 + b^2 + 2ab - 4 ab = a^2 + b^2 - 2ab = (a-b)^2 \ge 0\]

Donc :
\[4ab \le (a+b)^2\]

\subsubsection*{2.}
\begin{equation*}
	\begin{split}
		\frac{1}{8}a^2 + 2b^2 - ab &= \frac{1}{8}(a^2+16b^2-8ab) \\
		                           &= \frac{1}{8}(a-4b)^2 \\
		                           &\ge 0
	\end{split}
\end{equation*}	
Donc : \[ab \le\frac{1}{8}a^2 + 2b^2\]

\section*{Application 1.31}

Soit $f(x) = 9x+3$ ; $f$ est croissante sur $\mathbb{R}$ donc :

\begin{equation*}
	\begin{split}
		& \forall x \in [1 ; 2], f(1) \le f(x) \le f(2) \\
		\Longleftrightarrow & \forall x \in [1 ; 2], 12 \le 9x+3 \le 21 \\
	\end{split}
\end{equation*}

Soit $g(x)=x^2+2$ qui est croissante et positive sur $[1;2]$, donc :
\begin{equation*}
	\begin{split}
		                & \forall x \in [1 ; 2], g(1) \le g(x) \le g(2) \\
		\Longrightarrow & \forall x \in [1 ; 2], \frac{1}{g(2)} \le \frac{1}{g(x)} \le \frac{1}{g(1)} \\
		\implies        & \forall x \in [1 ; 2], \frac{1}{6} \le \frac{1}{x^2+5} \le \frac{1}{3} \\
	\end{split}
\end{equation*}

En faisant le produit des 2 inégalités à termes positifs précédentes, nous obtenons :
 
\begin{equation*}
	\begin{split}
		         & \forall x \in [1 ; 2], \frac{12}{6} \le \frac{9x + 3}{x^2+5} \le \frac{21}{3} \\
		\implies & \forall x \in [1 ; 2], 2 \le \frac{9x + 3}{x^2+5} \le 7 \\
	\end{split}
\end{equation*}
CQFD

\section*{Exercices}

\subsection*{Exercice 1.1}

\[ \forall n \in \mathbb{N}^*, \binom{n}{1} =\frac{n!}{1!(n-1)!} = \frac{n(n-1)!}{(n-1)!} = n\]

\subsection*{Exercice 1.2}

\subsubsection*{1}
\[ \binom{7}{3} = \frac{7!}{3!4!} = \frac{5 \times 6 \times 7}{6} = 35\]

\subsection*{2}

D'après la relation de Pascal nous avons :
\begin{equation*}
	\begin{split}
		\binom{6}{4} &= \binom{5}{4} + \binom{5}{3} \\
		             &= \binom{4}{4} + \binom{4}{3} + \binom{4}{3} + \binom{4}{2} \\
		             &= \binom{4}{2} + 2\binom{4}{3} + \binom{4}{4}
	\end{split}
\end{equation*}

On peut également calculer :
\[\binom{6}{4} = \frac{6!}{4!2!} = \frac{5 \times 6}{2} = 15\]

\begin{equation*}
	\begin{split}
		\binom{4}{2} + 2\binom{4}{3} + \binom{4}{4} &= \frac{4!}{2!2!} + 2\frac{4!}{3!1!} + 1 \\
		                                            &= 6 + 2 \times 4 + 1 \\
		                                            &= 15
	\end{split}
\end{equation*}


\section*{Exercice 1.3}

\begin{equation*}
	\begin{split}
		\forall n \in \mathbb{N}^*, \forall k \in \{1,\ldots, n\}, n\binom{n-1}{k-1} &= n \frac{(n-1)!}{(k-1)!(n-k)!} \\
		                                                                          &= \frac{n!}{\frac{k!}{k}(n-k)!} \\
		                                                                          &= k \frac{n!}{k!(n-k)!} \\
		                                                                          &= k \binom{n}{k}
	\end{split}
\end{equation*}

\section*{Exercice 1.4}
\begin{alignat*}{2}
	                    &        &   \binom{n}{1} + \binom{n}{2} + \binom{n}{3} &= 5n \\
	\Longleftrightarrow & \qquad & n + \frac{n(n-1)}{2} + \frac{n(n-1)(n-2)}{6} &= 5n \\
	\Longleftrightarrow &        & 1 + \frac{n-1}{2} + \frac{(n-1)(n-2)}{6}     &= 5 \\
	\Longleftrightarrow &        &                      6 + 3(n-1) + (n-1)(n-2) &= 30 \\
    \Longleftrightarrow &        &                      n^2 &= 25 \\
    \Longleftrightarrow &        &                      n   &= 5 \text{ dans $\mathbb{N}^*$}
\end{alignat*}

\section*{Exercice 1.5}	


\[2 + 4 + \cdots + 22 + 24 = \sum_{i=1}^{12}2i 
		                   = 2\sum_{i=1}^{12}i 
		                   = 2\frac{12 \times 13}{2} 
		                   = 156 \]

\section*{Exercice 1.6}	
\begin{equation*}
	\begin{split}
		\sum_{k=1}^{n}\ln\left( \frac{k+1}{k}\right)  &= \sum_{k=1}^{n}\left( \ln(k+1)-\ln(k)\right) \\
		                                             &=  \sum_{k=2}^{n+1}\ln(k)- \sum_{k=1}^{n}\ln(k) \\
		                                             &= \ln(n+1) - \ln(1) \\
		                                             &= \ln(n+1)
	\end{split}
\end{equation*}

\section*{Exercice 1.7}	
\begin{equation*}
	\begin{split}
		S &=\sum_{k=1}^{n}\left( \frac{1}{k} - \frac{1}{n+1-k}\right) \\
		  &=\sum_{k=1}^{n} \frac{1}{k} - \sum_{k=1}^{n}\frac{1}{n+1-k} \\
		  &=\sum_{k=1}^{n} \frac{1}{k} - \sum_{j=1}^{n}\frac{1}{j} \text{ avec $j=n+1-k$}\\
		  &=0
	\end{split}
\end{equation*}

\section*{Exercice 1.8}	
\begin{equation*}
	\begin{split}
		\sum_{k=0}^{n}\frac{k}{(k+1)!} &= \sum_{k=0}^{n}\frac{k + 1 - 1}{(k+1)!} \\
		                               &= \sum_{k=0}^{n}\frac{k + 1}{(k+1)!} - \sum_{k=0}^{n}\frac{1}{(k+1)!} \\
		                               &= \sum_{k=0}^{n}\frac{1}{k!} - \sum_{k=1}^{n+1}\frac{1}{k!} \\
		                               &= \frac{1}{0!}-\frac{1}{(n+1)!} \\
		                               &= 1 -\frac{1}{(n+1)!}
	\end{split}
\end{equation*}


\section*{1.9 Somme des carrés}

Soit la proposition $P(n)_{n \in \mathbb{N}}$ : « $ \sum_{k=0}^{n}k^2 = \frac{n(n+1)(2n+1)}{6}$ »

$P(0)$ est vraie. Supposons $P(N)$ vraie.

\begin{equation*}
	\begin{split}
		\sum_{k=0}^{N+1}k^2 &= \sum_{k=0}^{N}k^2 + (N+1)^2 \\
		                    &= \frac{N(N+1)(2N+1)}{6} + (N+1)^2 \\
		                    &= \frac{(N+1)\left[N(2N+1) + 6(N+1)\right]}{6} \\
		                    &= \frac{(N+1)(2N^2 + 7N +6)}{6} \\
		                    &= \frac{(N+1)(N+2)(2N + 3)}{6} \\
		                    &= \frac{(N+1)((N+1)+1)(2(N+1) + 1)}{6}
	\end{split}
\end{equation*}
$P(N+1)$ est donc vraie.

\section*{Exercice 1.10}
\begin{equation*}
	\begin{split}
		\sum_{k=0}^{n} k\binom{n}{k} &= 0 + \sum_{k=1}^{n-1} k\binom{n}{k} \\
		                             &= \sum_{k=1}^{n} n\binom{n-1}{k-1} \\
		                             &= n\sum_{k=0}^{n-1} \binom{n-1}{k} \\
		                             &= n 2^{n-1} \\
	\end{split}
\end{equation*}


\section*{Exercice 1.11}
\begin{equation*}
	\begin{split}
		\sum_{k=n+1}^{2n-1} \exp \left(\sin \left(\frac{k\pi}{2n}\right)\right)
			&= \sum_{k=n+1}^{2n-1} \exp \left(\sin \left(\pi - \frac{k\pi}{2n}\right)\right) \\
			&= \sum_{k=n+1}^{2n-1} \exp \left(\sin \left( \frac{(2n-k)\pi}{2n}\right)\right) \\
			&= \sum_{i=1}^{n-1} \exp \left(\sin \left( \frac{i\pi}{2n}\right)\right) \text{ avec $i = 2n-k$}
	\end{split}
\end{equation*}

\section*{Exercice 1.12}

On raisonne par récurrence et on utilisera le résultat de l'exercice 1.3.

\begin{equation*}
	\begin{split}
		\sum_{k=1}^{n} \frac{(-1)^{k+1}}{k}\binom{n}{k}
			&= 	\sum_{k=1}^{n-1} \frac{(-1)^{k+1}}{k}\binom{n}{k} + \frac{(-1)^{n+1}}{n} \\
			&= 	\sum_{k=1}^{n-1} \frac{(-1)^{k+1}}{k}\left[\binom{n-1}{k} + \binom{n-1}{k-1}\right] + \frac{(-1)^{n+1}}{n} \\
			&= 	\sum_{k=1}^{n-1} \frac{(-1)^{k+1}}{k}\binom{n-1}{k} + \sum_{k=1}^{n-1} \frac{(-1)^{k+1}}{k}\binom{n-1}{k-1} + \frac{(-1)^{n+1}}{n} \\
			&= 	\sum_{k=1}^{n-1} \frac{1}{k}\binom{n-1}{k} + \sum_{k=1}^{n-1} \frac{(-1)^{k+1}}{k}\binom{n-1}{k-1} + \frac{(-1)^{n+1}}{n} \text{ par récurrence} \\
			&= 	\sum_{k=1}^{n-1} \frac{1}{k} + \sum_{k=1}^{n-1} \frac{(-1)^{k+1}}{n}\binom{n}{k} + \frac{(-1)^{n+1}}{n} \text{ d'après exercice 1.3} \\
			&= 	\sum_{k=1}^{n-1} \frac{1}{k} - \frac{1}{n} \sum_{k=1}^{n-1} (-1)^{k}\binom{n}{k} + \frac{(-1)^{n+1}}{n} \\
			&= 	\sum_{k=1}^{n-1} \frac{1}{k} - \frac{1}{n} \sum_{k=0}^{n} 1^{n-k}(-1)^{k}\binom{n}{k} + \frac{1}{n} + \frac{(-1)^{n}}{n} + \frac{(-1)^{n+1}}{n} \\
			&= 	\sum_{k=1}^{n-1} \frac{1}{k} - \frac{1}{n} (1-1)^n + \frac{1}{n} \\
			&= 	\sum_{k=1}^{n} \frac{1}{k}
	\end{split}
\end{equation*}

\section*{Exercice 1.13}

\subsubsection*{a}
\begin{equation*}
	\begin{split}
		\sum_{0 \le i,j \le n} ij &= \sum_{i=0}^{n} \sum_{j=0}^{n} ij \\
		&= \sum_{i=0}^{n} \left(i \sum_{j=0}^{n} j \right)\\
		&= \sum_{j=0}^{n} j \times \sum_{i=0}^{n} i \\
		&= \left(\sum_{i=0}^{n} i \right)^2 \\
		&= \frac{n^2(n+1)^2}{4}
	\end{split}
\end{equation*}

\subsubsection*{b}
\begin{equation*}
	\begin{split}
		\sum_{0 \le i,j \le n} a^{i+j} &= \sum_{i=0}^{n} \sum_{j=0}^{n} a^i a^j \\
			&= \sum_{i=0}^{n} \left(a^i \sum_{j=0}^{n} a^j \right)\\
			&= \sum_{j=0}^{n} a^j \times \sum_{i=0}^{n} a^ji \\
			&= \left(\sum_{i=0}^{n} a^i \right)^2 \\
			&= \begin{cases}
				\left(\frac{a^{n+1}-1}{a-1}\right)^2 & \text{si $a\neq 1$}\\
				(n+1)^2                                  & \text{si $a= 1$}
				\end{cases}          
	\end{split}
\end{equation*}

\subsubsection*{c}
\begin{equation*}
	\begin{split}
		\sum_{0 \le i,j \le n} (i+j) &= \sum_{i=0}^{n} \sum_{j=0}^{n} (i+j) \\
		&= \sum_{i=0}^{n} \left(i(n+1) + \sum_{j=0}^{n} j \right)\\
		&= (n+1) \sum_{i=0}^{n} i + (n+1)\sum_{j=0}^{n} j  \\
		&= 2(n+1) \sum_{i=0}^{n} \\
		&= n(n+1)^2
	\end{split}
\end{equation*}

\subsubsection*{d}
\begin{equation*}
	\begin{split}
		\sum_{0 \le i,j \le n} (i+j)^2 &= \sum_{i=0}^{n} \sum_{j=0}^{n} (i^2+2ij+j^2) \\
		&= \sum_{i=0}^{n} \left((n+1)i^2 + \sum_{j=0}^{n} j^2 \right) + \sum_{0 \le i,j \le n} ij\\
		&= (n+1) \sum_{i=0}^{n} i^2 + (n+1)\sum_{j=0}^{n} j^2 + \frac{n^2(n+1)^2}{2}  \\
		&= 2(n+1) \sum_{i=0}^{n} i^2 + \frac{n^2(n+1)^2}{2} \\
		&= 2(n+1)\frac{n(n+1)(2n+1)}{6} + \frac{n^2(n+1)^2}{2} \\
		&= \frac{n(n+1)^2(4n+2+3n)}{6} \\
		&= \frac{n(n+1)^2(7n+2)}{6}
	\end{split}
\end{equation*}

\subsubsection*{e}

\begin{equation*}
	\begin{split}
		\sum_{0 \le i, j \le n} \min(i,j)
		    &= \sum_{i=0}^{n} \left( \sum_{j=0}^{i} j + \sum_{j=i}^{n}i\right) \\
			&= \sum_{i=0}^{n} \left( \sum_{j=0}^{i} j +(n-i)i\right)  \\
			&=\sum_{i=0}^{n} \left( \frac{i(i+1)}{2} + ni-i^2\right) \\
			&=\sum_{i=0}^{n} \left( \frac{2n+1}{2}i -\frac{i^2}{2}\right) \\
			&= \frac{2n+1}{2} \frac{n(n+1)}{2} - \frac{n(n+1)(2n+1)}{12} \\
			&= \frac{n(n+1)(2n+1)}{6}
	\end{split}
\end{equation*}

\section*{Exercice 14}

\begin{equation*}
	\begin{split}
		\sum_{k=1}^{n} \sum_{i=1}^{n}\ln(i^k)
			&=  \sum_{k=1}^{n} \sum_{i=1}^{n} k\ln(i)\\
			&=  \sum_{k=1}^{n} k \times \sum_{i=1}^{n} \ln(i)\\
			&= \frac{n(n+1)}{2} \ln\left( \prod_{i=1}^{n}i\right) \\
			&= \frac{n(n+1)}{2} \ln\left( n! \right)
	\end{split}
\end{equation*}


\section*{Exercice 15}

\begin{equation*}
	\begin{split}
		\sum_{k=1}^{n} k2^k
		&=  \sum_{k=1}^{n}\left(2^k \sum_{l=1}^{k}1\right)\\
		&=  \sum_{k=1}^{n}\left(\sum_{l=1}^{k} 2^k\right)\\
		&=  \sum_{k=1}^{n}\sum_{l=1}^{k} 2^k\\
	\end{split}
\end{equation*}

Donc :

\begin{equation*}
	\begin{split}
		\sum_{k=1}^{n} k2^k
			&= \sum_{k=1}^{n}\sum_{l=1}^{k} 2^k\\
			&= \sum_{1 \le l \le k \le n} 2^k\\
			&= \sum_{l=1}^{n}\sum_{k=l}^{n} 2^k\\
			&= \sum_{l=1}^{n} \left(2^{n+1} - 2^l\right) \text{ (somme d'une suite géométrique)}\\
			&= n 2^{n+1} - \sum_{l=1}^{n} 2^l \\
			&= n 2^{n+1} - (2^{n+1}-2)  \text{ (somme d'une suite géométrique)}\\
			&= (n-1)2^{n+1} +2
	\end{split}
\end{equation*}

\section*{Exercice 16}

\begin{equation*}
	\begin{split}
		\sum_{0\le i \le j \le n} j
		&= \sum_{j=0}^{n} \sum_{i=0}^{j} j \\
		&= \sum_{j=0}^{n} j(j+1) \\
		&= \sum_{j=0}^{n} (j + j^2) \\
		&= \frac{n(n+1)}{2} + \frac{n(n+1)(2n+1)}{6} \\
		&= \frac{n(n+1)(2n+4)}{6} \\
		&= \frac{n(n+1)(n+2)}{3} \\
	\end{split}
\end{equation*}

\begin{equation*}
	\begin{split}
		\sum_{0\le i, j \le n} i
		&= \sum_{j=0}^{n} \sum_{i=0}^{j} i \\
		&= \sum_{j=0}^{n} \frac{j(j+1)}{2} \\
		&= \frac{1}{2} \sum_{0\le i \le j \le n} j \\
		&= \frac{n(n+1)(n+2)}{6} 
	\end{split}
\end{equation*}

\section*{Exercice 17}

\subsubsection*{a}
\begin{equation*}
	\begin{split}
		\sum_{1\le j \le i \le n} \binom{i}{j}
		&= \sum_{i=1}^{n} \sum_{j=1}^{i} \binom{i}{j} \\
		&= \sum_{i=1}^{n} (2^i -1)\\
		&= 2^{n+1} - 2 - n
	\end{split}
\end{equation*}

\subsubsection*{b}
\begin{equation*}
	\begin{split}
		\sum_{1\le j \le i \le n} \frac{1}{i}
		&= \sum_{i=1}^{n} \sum_{j=1}^{i} \frac{1}{i} \\
		&= \sum_{i=1}^{n} \frac{i}{i} \\
		&= n
	\end{split}
\end{equation*}

\subsubsection*{c}
\begin{equation*}
	\begin{split}
		\sum_{1\le i \le j \le n} \frac{i}{j}
		&= \sum_{j=1}^{n} \sum_{i=1}^{j} \frac{i}{j} \\
		&= \sum_{j=1}^{n} \frac{j(j+1)}{2j} \\
		&= \frac{1}{2}\sum_{j=1}^{n} (j+1) \\
		&= \frac{1}{2} \left(\frac{n(n+1)}{2} + n\right) \\
		&= \frac{n(n+3)}{4}
	\end{split}
\end{equation*}

\section*{Exercice 18}

\subsubsection*{1}

\begin{displaymath}
	x < y \Rightarrow x + y < y + y \Rightarrow \frac{x+y}{2} < y \Rightarrow m <y
\end{displaymath}

\subsubsection*{2}

Inégalité de Young avec $\sqrt{x}$ et $\sqrt{y}$ :

\begin{displaymath}
	\sqrt{x} \sqrt{y} < \frac{\sqrt{x}^2 + \sqrt{y}^2}{2} \Rightarrow g<m
\end{displaymath}

\subsubsection*{3}

\begin{displaymath}
	h = \frac{2xy}{x+y} = \frac{g^2}{m} < \frac{gm}{m} \Rightarrow h < g
\end{displaymath}

\subsubsection*{4}

\begin{displaymath}
	h = \frac{2xy}{x+y} > \frac{2xy}{y+y} \Rightarrow h > x
\end{displaymath}

\section*{Exercice 19}
\begin{displaymath}
	(1+x)^n = \sum_{k=0}^{n}\binom{n}{k}x^k = 1 + nx + \sum_{k=2}^{n}\binom{n}{k}x^k \geq 1 + nx
\end{displaymath}

\section*{Exercice 20}

\subsubsection*{1}
\begin{equation*}
	\begin{split}
	&\begin{cases}
		\forall x \in [1, 2], 5 \le 2x+3 \le 7 \\
		\forall x \in [1, 2], 7 \le x^2+6 \le 10 \\
	\end{cases} \\
	\Rightarrow
	&\begin{cases}
		\forall x \in [1, 2], 5 \le 2x+3 \le 7 \\
		\forall x \in [1, 2], \frac{1}{10} \le \frac{1}{x^2+6} \le \frac{1}{7} \\
	\end{cases} \\
	\Rightarrow
	& \forall x \in [1, 2], \frac{5}{10} \le \frac{2x+3}{x^2+6} \le \frac{7}{7} \\
	\Rightarrow
	&\forall x \in [1, 2], \frac{1}{2} \le \frac{2x+3}{x^2+6} \le 1
	\end{split}
\end{equation*}

\subsubsection*{2}
\begin{equation*}
	\begin{split}
		&\begin{cases}
			\forall x \in [1, 2], 2 \le 2x \le 4 \\
			\forall x \in [1, 2], -4 \le x^2-5 \le -1 \\
		\end{cases} \\
		\Rightarrow
		&\begin{cases}
			\forall x \in [1, 2], 2 \le 2x \le 4 \\
			\forall x \in [1, 2], \frac{1}{4} \le \frac{-1}{x^2-5} \le 1 \\
		\end{cases} \\
		\Rightarrow
		& \forall x \in [1, 2], \frac{2}{4} \le \frac{-2x}{x^2-5} \le \frac{4}{1} \\
		\Rightarrow
		& \forall x \in [1, 2], -4 \le \frac{2x}{x^2-5} \le \frac{-1}{2} \\
	\end{split}
\end{equation*}

\end{document}  