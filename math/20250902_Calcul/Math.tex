% XeLaTeX can use any Mac OS X font. See the setromanfont command below.
% Input to XeLaTeX is full Unicode, so Unicode characters can be typed directly into the source.

% The next lines tell TeXShop to typeset with xelatex, and to open and save the source with Unicode encoding.

%!TEX TS-program = xelatex
%!TEX encoding = UTF-8 Unicode

\documentclass[12pt]{smfbook}
\usepackage{geometry}                % See geometry.pdf to learn the layout options. There are lots.
\geometry{a4paper}                   % ... or a4paper or a5paper or ... 
%\geometry{landscape}                % Activate for for rotated page geometry
%\usepackage[parfill]{parskip}    % Activate to begin paragraphs with an empty line rather than an indent
\usepackage{graphicx}
\usepackage{amssymb}

% Will Robertson's fontspec.sty can be used to simplify font choices.
% To experiment, open /Applications/Font Book to examine the fonts provided on Mac OS X,
% and change "Hoefler Text" to any of these choices.

\usepackage{fontspec,xltxtra,xunicode}
\defaultfontfeatures{Mapping=tex-text}
\setromanfont[Mapping=tex-text]{Hoefler Text}
\setsansfont[Scale=MatchLowercase,Mapping=tex-text]{Gill Sans}
\setmonofont[Scale=MatchLowercase]{Andale Mono}

\title{Math - Exercice}
\author{Serge RAVEZ}
%\date{}                                           % Activate to display a given date or no date

\begin{document}
\maketitle

% For many users, the previous commands will be enough.
% If you want to directly input Unicode, add an Input Menu or Keyboard to the menu bar 
% using the International Panel in System Preferences.
% Unicode must be typeset using a font containing the appropriate characters.
% Remove the comment signs below for examples.

% \newfontfamily{\A}{Geeza Pro}
% \newfontfamily{\H}[Scale=0.9]{Lucida Grande}
% \newfontfamily{\J}[Scale=0.85]{Osaka}

% Here are some multilingual Unicode fonts: this is Arabic text: {\A السلام عليكم}, this is Hebrew: {\H שלום}, 
% and here's some Japanese: {\J 今日は}.


\chapter{Calcul algébrique}

\section*{Application 1.2}

\subsection*{1}
\begin{equation*}
\begin{split}
	(a+b+c)^{2} &= (a+b+c)(a+b+c) \\
	&= a^{2}+ab+ac+ba+b^{2}+bc+ca+cb+c^{2}\\
	&=a^{2}+b^{2}+c^{2} + 2(ab+ac+bc)
\end{split}	
\end{equation*}


\subsection*{2}

\begin{equation*}
	\begin{split}
		10002 \times 99998 &= (10^4 +2)(10^4 - 2) \\
		                                   &= 10^8 - 4 \\
		                                   &= 99 999 996
	\end{split}
\end{equation*}

\begin{equation*}
	\begin{split}
		100001^2 &= (10^5 + 1)^2 \\
		&= 10^{10} + 2 \times 10^5 + 1 \\
		&= 10000200001
	\end{split}
\end{equation*}



\subsection*{3}

\begin{equation*}
	\begin{split}
		(2x-5)^2 - (2x - 9)^2 &= (2x-5+2x-9)(2x-5-2x+9) \\
		&= 4(4x-14) \\
		&= 8(2x-7)
	\end{split}
\end{equation*}



\section*{Application 1.8}

\subsection*{1}

\begin{displaymath}
	\binom{n}{0}= \frac{n!}{0!(n-0)!}= \frac{n!}{1 \times n!}=1
\end{displaymath}

\subsection*{2}
\begin{displaymath}
	\binom{n}{n}= \frac{n!}{(n-n)!n!}= \frac{n!}{0!n!}=1
\end{displaymath}


\subsection*{3}
\begin{displaymath}
	\binom{n}{k}= \frac{n!}{(n-k)!k!}= \frac{n!}{(n-k)!(n-(n-k))!}=\binom{n}{n-k}
\end{displaymath}



\section*{Proposition 1.9 - Raisonnement de dénombrement}

Soit $E$ un ensemble de $n$ éléments et $e$ un de ces éléments.

Le nombre de combinaisons de $k$ éléments contenant $e$ est $N_e = \binom{n-1}{k-1}$ (il faut choisir $k-1$ éléments dans les $n-1$ restants)

Le nombre de combinaisons de $k$ éléments ne contenant pas $e$ est $N_{\bar{e}} = \binom{n-1}{k}$ (il faut choisir $k$ éléments dans les $n-1$ restants)

Donc :
\begin{displaymath}
\binom{n}{k}= N_e + N_{\bar{e}} =  \binom{n-1}{k-1}+ \binom{n-1}{k}
\end{displaymath}


\section*{Application 1.10}

Soit la proposition $P(n)_{n \in \mathbb{N}^*}$ : « $\forall  k \in \left\lbrace 1,\ldots, n\right\rbrace, \binom{n}{k} \in \mathbb{N}^*$ »

$P(1)$ est vraie

Supposons que  $P(N)$ avec $N \in \mathbb{N}^*$ est vrai ; on a :

\begin{equation*}
	\begin{cases}
		\binom{N+1}{0} = 1 \Rightarrow \binom{N+1}{0} \in \mathbb{N}^* \\
		\forall k \in \left\lbrace 1, ... , N\right\rbrace , \binom{N+1}{k} = \binom{N}{k} + \binom{N}{k-1} \in \mathbb{N}^* \text{somme de 2 entiers naturels non nuls}\\
		\binom{N+1}{N+1} = 1  \Rightarrow \binom{N+1}{0} \in \mathbb{N}^*
	\end{cases}
\end{equation*}

$P(N+1)$ est donc vraie


\section*{Proposition 1.16}

Si la raison est différente de $1$, la somme d'une suite géométrique est : « le suivant moins le premier sur la raison moins $1$».

\section*{Application 1.22}

\subsection*{1} Somme de 7 éléments consécutifs d'une suite arithmétique
\begin{displaymath}
	2 + 3 + 4 + 5 + 6 + 7 + 8 = \frac{(8+2)\times 7}{2} = 35
\end{displaymath}

\subsection*{2} Somme de termes consécutifs d'une suite géométrique
\begin{displaymath}
\sum_{k = 0}^{4} 3^k = \frac{3^5-1}{3-1}=\frac{243-1}{2}=121
\end{displaymath}


\subsection*{3} Sommes de coefficients binomiaux
\begin{displaymath}
	\sum_{k = 0}^{n} \binom{n}{k} = \sum_{k = 0}^{n} \binom{n}{k} 1^k . 1^{n-k} = (1+1)^n = 2^n
\end{displaymath}
\begin{displaymath}
	\sum_{k = 0}^{n} (-1)^k \binom{n}{k} = \sum_{k = 0}^{n} \binom{n}{k} (-1)^k . 1^{n-k} = (1-1)^n = 0
\end{displaymath}


\section*{Application 1.27 - Inégalités}

\subsection*{1} Changement de sens
\begin{displaymath}
	\begin{split}
                     x &\leq y \\
\Rightarrow x - y &\leq y - y   \text{ d'après $(b)$}\\
\Rightarrow x - y - x &\leq 0 - x  \text{ d'après $(b)$}\\
\Rightarrow -y& \leq -x
	\end{split}
\end{displaymath}


\subsection*{2} Addition d'inégalités

L'addition d'un réel à une inégalité $(b)$ permet de dire :
\begin{displaymath}
	\begin{split}
x \leq y & \Rightarrow x  + z\leq y + z \\
z \leq t & \Rightarrow z + y \leq t + y
	\end{split}
\end{displaymath}

Donc d'après la transitivité $(c)$  : $ x  + z \leq y + t$ 

\subsection*{3} Soustraction d'inégalités
 
Changement de sens : $z \leq t \Rightarrow -t \leq -z$ (cf. ci-dessous)

Addition d'inégalités : $x - t \leq y - z$ 

\subsection*{4} Produit d'un réel négatif avec une inégalités

$ z < 0 \Rightarrow -z > 0 \Rightarrow x(-z) \leq y(-z)$ d'après (d)

$ \Rightarrow xz \geq yz$ d'après la première proposition

\subsection*{5} Produit d'inégalités à termes positifs

\begin{displaymath}
	\begin{split}
	x \leq y  &\wedge z\leq t\\
	\Rightarrow xz \leq yz &\wedge zy \leq ty \text{ d'après $(d)$} \\
	\Rightarrow xz &\leq yt \text{ d'après $(c)$}
	\end{split}
\end{displaymath}


\end{document}  