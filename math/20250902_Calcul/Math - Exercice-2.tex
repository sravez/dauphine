% XeLaTeX can use any Mac OS X font. See the setromanfont command below.
% Input to XeLaTeX is full Unicode, so Unicode characters can be typed directly into the source.

% The next lines tell TeXShop to typeset with xelatex, and to open and save the source with Unicode encoding.

%!TEX TS-program = xelatex
%!TEX encoding = UTF-8 Unicode

\documentclass{report}
\usepackage[french]{babel}
\usepackage{amsmath}
\usepackage{geometry}                % See geometry.pdf to learn the layout options. There are lots.
\geometry{a4paper,hmargin=2cm,vmargin=1.5cm,includeheadfoot}                   % ... or a4paper or a5paper or ... 
%\geometry{landscape}                % Activate for for rotated page geometry
%\usepackage[parfill]{parskip}    % Activate to begin paragraphs with an empty line rather than an indent
\usepackage{graphicx}
\usepackage{amssymb}
\usepackage{tkz-tab}
\usepackage{pgfplots}
\pgfplotsset{compat=1.18}


%\usepackage{fontspec,xltxtra,xunicode}
%\defaultfontfeatures{Mapping=tex-text}
%\setromanfont[Mapping=tex-text]{Hoefler Text}
%\setsansfont[Scale=MatchLowercase,Mapping=tex-text]{Gill Sans}
%\setmonofont[Scale=MatchLowercase]{Andale Mono}

\title{Math - Exercices}
\author{Poppy RAVEZ}
%\date{}                                           % Activate to display a given date or no date

\begin{document}


\chapter*{2 - Fonctions réelles usuelles}


\subsection*{Application 2.3}

\subsubsection*{a. $\forall x \in \mathbb{R}, x \leq |x|$}

\begin{displaymath}
	\begin{cases}
		\forall x \in \mathbb{R}^{+}, |x| = x \geq x \\
		\forall x \in \mathbb{R}^{-}_{*}, |x| = -x > 0 > x
	\end{cases}
\end{displaymath}
Donc $\forall x \in \mathbb{R}, x \leq |x|$

\subsubsection*{b. $\forall x \in \mathbb{R}, |x| = 0 \Longleftrightarrow x = 0$}

Par définition : $x = 0 \Longrightarrow |x| = 0$.

Si $|x| = 0$, les solutions  vérifient :

\begin{displaymath}
	\begin{cases}
		x = |x| = 0 \text{ si } x \in \mathbb{R}^{+}\\
		-x = |x| = 0 \text{ si } x \in \mathbb{R}^{-}_{*}
	\end{cases}
\end{displaymath}

Donc $x=0$

L'implication réciproque $|x|=0, \Longrightarrow x=0$ est donc vraie ce qui démontre l'équivalence.

\subsubsection*{c. $\forall (x,y) \in \mathbb{R}^2, |xy| = |x|.|y|$}

\begin{tabular}{|c|c|c|}
	\hline
	$x$                  & $y$                  & $|xy|$\\
	\hline
	$\mathbb{R}^{+}$     & $\mathbb{R}^{+}$     & $|xy| = xy = |x|.|y|$ \\
	\hline
	$\mathbb{R}^{+}$     & $\mathbb{R}^{-}_{*}$ & $|xy| = -xy = x(-y) = |x|.|y|$ \\
	\hline
	$\mathbb{R}^{-}_{*}$ & $\mathbb{R}^{+}$     & $|xy| = -xy = (-x)y = |x|.|y|$ \\
	\hline
	$\mathbb{R}^{-}_{*}$ & $\mathbb{R}^{-}_{*}$ & $|xy| = xy =(-x)(-y) = |x|.|y|$ \\
	\hline
\end{tabular}

CQFD

Dans l'absolu et pour coller à la définition, il faudrait distinguer les cas $xy<0$ et $xy=0$ quand les deux variables n'ont pas
le même signe.


\subsubsection*{d. $\forall x \in \mathbb{R}, \forall z \in \mathbb{R}^{+}, |x| \leq z \Longleftrightarrow -z \leq x \leq z$}

Soit $z \in \mathbb{R}^{+}$ (et donc $-z \leq 0$).

Si $x \in \mathbb{R}^{+}$ (et donc $|x|=x \geq 0$):

\begin{itemize}
	\item $|x| \leq z \Longrightarrow -z \leq 0 \leq x \leq z \Longrightarrow -z \leq x \leq z$
	\item $-z \leq x \leq z \Longrightarrow -z \leq 0 \leq x \leq z \Longrightarrow |x| \leq z$
\end{itemize}

Ce qui démontre l'équivalence sur $\mathbb{R}^{+}$ ; l'équivalence sur $\mathbb{R}^{-}$ en est la conséquence
en remplaçant $x$ par $-x$ (les deux inégalités étant invariantes par changement de signe).

\subsection*{Application 2.12}

\subsubsection*{1. $\exp(-\frac{1}{x^2})$}
\begin{displaymath}
	\lim_{x \rightarrow 0} -\frac{1}{x^2} =-\infty \Longrightarrow 	\lim_{x \rightarrow 0} exp(-\frac{1}{x^2})= 0 \text{ car } \lim_{x \rightarrow -\infty} x = 0
\end{displaymath}

\subsubsection*{2.}
\begin{displaymath}
	\lim_{x \rightarrow +\infty} (1 + e^x + x)= +\infty
\end{displaymath}

\subsubsection*{3.}
\begin{displaymath}
	\lim_{x \rightarrow +\infty} \frac{e^x+1}{e^x-1}= \lim_{x \rightarrow +\infty} \frac{1+e^{-x}}{1-e^{-x}} = 1 \text{ car }
	\lim_{x \rightarrow +\infty} e^{-x} = 0
\end{displaymath}


\subsection*{Application 2.16}

\begin{equation*}
	\begin{split}
		\ln(4-\sqrt{3}) + \ln(4+\sqrt{3})
		&= \ln ((4-\sqrt{3})(4+\sqrt{3}))\\
		&= \ln (4^2-\sqrt{3}^2) \\
		&= \ln 13
	\end{split}
\end{equation*}

\begin{equation*}
	\begin{split}
		\ln(\sqrt{x+1})
		&= 	\ln((x+1)^{1/2})\\
		&= \frac{\ln(x+1)}{2}
	\end{split}
\end{equation*}

\subsection*{Application 2.17}
D'après la définition de la dérivée d'une fonction :
\begin{displaymath}
\lim_{x \rightarrow 0} \frac{\ln(1+x)}{x} = \lim_{x \rightarrow 0} \frac{\ln(1+x) - \ln(1)}{x} = \ln'(1) = \frac{1}{1} = 1
\end{displaymath}


\subsection*{Application 2.24}


\subsection*{Application 2.32}
\begin{equation*}
	\begin{split}
		\forall x \in \mathbb{R}^{+}_{*}, \left(  x^{\frac{1}{x}} \right)^{x^2+2x}
		&= x^{\frac{x^2+2x}{x}} \\
		&= x^{x+2}
	\end{split}
\end{equation*}


\subsection*{Application 2.34 - Dérivabilité de $p_{\alpha}$ en $0$ avec $\alpha \geq 1$}

$p_{\alpha}$ est définie sur $\mathbb{R}^{+}_{*}$ par $p_{\alpha}(x) = \exp(\alpha \ln(x))$, cependant :
\begin{displaymath}
	\lim_{x\rightarrow 0^{+}} \ln(x) = -\infty \Longrightarrow\forall \alpha>0, \lim_{x\rightarrow 0^{+}} p_{\alpha}(x) = 0
\end{displaymath}

On peut donc, si $\alpha > 0 0$, définir par continuité la fonction $p_{\alpha}$ en $0$ avec la valeur nulle.

On a alors :
\begin{equation*}
	\begin{split}
		\lim_{h\rightarrow 0^{+}} \frac{p_{\alpha}(h) - 0}{h}
		&= \lim_{h\rightarrow 0^{+}} \frac{e^{\alpha \ln(h)}}{h} \\
		&= \lim_{h\rightarrow 0^{+}} \frac{e^{\alpha \ln(h)}}{e^{\ln(h)}} \\
		&= \lim_{h\rightarrow 0^{+}} e^{(\alpha-1) \ln(h)} \\
		&=0 \text{ si $\alpha >1$ car alors $\lim_{h\rightarrow 0^{+}} (\alpha-1) \ln(h) = -\infty$}
	\end{split}
\end{equation*}

La fonction $p_{\alpha}$ prolongée en $0$ avec $p_{\alpha}(0)=0$ est donc dérivable à droite en $0$ si $\alpha>1$
avec $p'_{\alpha}(0)=0$.


\subsection*{Application 2.41 - Limite en $0$ de $x^x$}

\begin{equation*}
	\begin{split}
		\lim_{x\rightarrow 0^{+}} x^x
		&= \lim_{x\rightarrow 0^{+}} e^{x\ln(x)} \\
	    &= 1 \text{ car  $\lim_{x\rightarrow 0^{+}} x\ln(x) = 0$ d'après la proposition 2.39}
	\end{split}
\end{equation*}


\subsection*{Exercice 2.1}

\begin{equation*}
	\begin{split}
		\forall x \in \mathbb{R}, \lim_{h\rightarrow 0} \dfrac{\dfrac{1}{1+(x+h)^2}-\dfrac{1}{1+x^2}}{h}
		&= \lim_{x\rightarrow 0} \dfrac{1+x^2-1-(x+h)^2}{h(1+(x+h)^2)(1+x^2)} \\
		&= \lim_{x\rightarrow 0} \dfrac{-2xh-h^2}{h(1+(x+h)^2)(1+x^2)} \\
		&= \lim_{x\rightarrow 0} \dfrac{-2x-h}{(1+(x+h)^2)(1+x^2)} \\
		&= \frac{-2x}{(1+x^2)^2} \\
	\end{split}
\end{equation*}

Car la limite du dénominateur est un réel strictement positif.

La fonction $f$ est donc dérivable sur $\mathbb{R}$ et $f'(x) = \frac{-2x}{(1+x^2)^2}$

On constate de manière évidente que $f$ est paire, strictement positive, atteint son maximum en $0$ et que :
\begin{displaymath}
	\lim_{x \rightarrow \pm \infty} f(x) = 0^{+}
\end{displaymath}


On en déduit le tableau de variation suivant :

\begin{tikzpicture}
	\tkzTabInit{$x$ / 1 , $f'(x)$ / 1, $f(x)$/2}{$-\infty$ ,$0$, $+\infty$}
	\tkzTabLine{, + , 0 , - , }
	\tkzTabVar{-/$0$, +/$1$, -/$0$}
\end{tikzpicture}

Recherchons les points d'inflexion (où elle traverse sa tangente) caractérisés par $f''(x)=0$ :

\begin{equation*}
	\begin{split}
		f''(x) &= \dfrac{-2(1+x^2)^2-2(-2x)(2x)(1+x^2)}{(1+x^2)^4} \\
		       &= \dfrac{-2(1+x^2)+8x^2}{(1+x^2)^3} \\
		       &= 2\dfrac{3x^2-1}{(1+x^2)^3} 
	\end{split}
\end{equation*}

Les points d'inflexion sont donc $\binom{\pm\frac{1}{\sqrt{3}}}{\frac{3}{4}}$.

\begin{tikzpicture}
	\begin{axis}[
		axis lines=center,
		height=5cm,
		width=10cm,
		xlabel={$x$},
		ylabel={$f(x)$},
		xmin=-4, xmax= 4,
		ymin=0, ymax=1.1,
		]
		\addplot[blue,samples=81]{1/(1+x^2)};
	\end{axis}	
\end{tikzpicture}


\subsection*{Exercice 2.2}

$f$ est impaire, continue et dérivable sur $\mathbb{R}$.

$f(0) = 0$.

\begin{displaymath}
\lim_{x \rightarrow -\infty} f(x) = 0^{-}
\end{displaymath}

\begin{displaymath}
	\lim_{x \rightarrow +\infty} f(x) = 0^{+}
\end{displaymath}

On peut remarquer que :
\begin{displaymath}
	f_n\left(\frac{1}{\sqrt{n}} x \right) = \frac{1}{\sqrt{n}} f_1(x).
\end{displaymath}

La représentation graphique de $f_n$ est donc la transformée de celle de $f_1$
par l'homothétie ayant pour centre l'origine et de rapport $\frac{1}{\sqrt{n}}$.
L'homothétie conservant les directions (et donc les sens de variations, extrema et
points d'inflexion), ll suffit d'étudier $f_1$.

Cette caractéristique, si elle nous échappe initialement, deviendra évidente durant
l'étude de la fonction lorsque l'on constatera l'alignement des points particuliers
(extrema et points d'inflexion).

\subsubsection*{Variations}

\begin{displaymath}
	f'(x) = \frac{1+nx^2-2nxx}{(1+nx^2)^2}=\frac{1-nx^2}{(1+nx^2)^2}
\end{displaymath}

$f'(0) = 1$ : toutes les courbes en la même tangente à l'origine

Les extrema (où $f'(x)=0$) sont donc $\pm\frac{1}{\sqrt{n}}$

$f(\frac{1}{\sqrt{n}}) = \frac{1}{2\sqrt{n}}$

On constate que $\frac{f(1/\sqrt{n})}{1/\sqrt{n}} = \frac{1}{2}$, les extrema sont donc tous sur la droite
d'équation $y=\frac{x}{2}$.

Le tableau de variation est donc :

\begin{tikzpicture}
	\tkzTabInit{$x$ / 1 , $f'(x)$ / 1, $f(x)$/2}{$-\infty$ ,$-\frac{1}{\sqrt(n)}$, $+\frac{1}{\sqrt(n)}$, $+\infty$}
	\tkzTabLine{, - , 0 , +, 0, - , }
	\tkzTabVar{+/$0^{-}$, -/$-\frac{1}{2\sqrt{n}}$, +/$\frac{1}{2\sqrt{n}}$, -/$0^{+}$}
	\tkzTabVal{2}{3}{0.5}{0}{0}
\end{tikzpicture}

\subsubsection*{Points d'inflexion}

\begin{displaymath}
	f''(x) = \frac{-2nx(1+nx^2)^2-4nx(1-nx^2)(1+nx^2)}{(1+nx^2)^4}
	       =-2nx\frac{1+nx^2+2(1-nx^2)}{(1+nx^2)^3}
	       = -2nx\frac{3-nx^2}{(1+nx^2)^3}
\end{displaymath}

Chaque courbe présente donc 3 points d'inflexion alignés en $-\sqrt{\frac{3}{n}}$, $0$ et $\sqrt{\frac{3}{n}}$.

On constate que $\frac{f(\sqrt{\frac{3}{n}})}{\sqrt{\frac{3}{n}}} = \frac{1}{4}$ et que $f'\left({\pm \sqrt{\frac{3}{n}}}\right)=-1/8$, ce qui montre que tous les points d'inflexion sont sur la droite
$y=x/4$ et que les tangentes en ces points sont toutes parallèles.

\subsubsection*{Représentation graphique}

\begin{tikzpicture}
	\begin{axis}[
		axis lines=center,
		height=5cm,
		width=10cm,
		xlabel={$x$},
		ylabel={$f(x)$},
		xmin=-4, xmax= 4,
		ymin=-0.6, ymax=0.6,
		legend pos= north west,
		]
		\addplot[blue,samples=81]{x/(1+x^2)};
		\addlegendentry{$n=1$}
		\addplot[red,samples=81]{x/(1+2*x^2)};
		\addlegendentry{$n=2$}
		\addplot[green,samples=81]{x/(1+3*x^2)};
		\addlegendentry{$n=3$}
	\end{axis}	
\end{tikzpicture}

\subsection*{Exercice 2.3}

\subsubsection*{1.}

\begin{equation*}
	\begin{split}
		|\sin^{12}(x) - 2\cos^9(x)| &\leq |\sin^{12}(x)|+ |2\cos^9(x)| \text{ (inégalité triangulaire)} \\
		                         &\leq 1 + 2 \\
		                         & \leq 3
	\end{split}
\end{equation*}

\subsubsection*{2.}

\begin{equation*}
	\begin{split}
		|1-x^{12} +4x^3| &\leq |1- x^{12}|+ |4x^3| \text{ (inégalité triangulaire)} \\
		&\leq 1 + 4 \text{ pour $x \in [-1, 1]$}\\
		& \leq 5
	\end{split}
\end{equation*}


\subsection*{Exercice 2.4}

\subsubsection*{(a)}

\begin{alignat*}{2}
&                    & |x+1| &=2 \\
&\Longleftrightarrow &(x+1=2 \wedge x+1 \geq 0) &\vee (-x-1=2 \wedge x+1 \leq 0) \\
&\Longleftrightarrow & x=1 &\vee x=-3 \\
\end{alignat*}

Donc : $S=\left\lbrace -3 ; 1 \right\rbrace$

\subsubsection*{(b)}

\begin{alignat*}{2}
	&                    & |x+3| &\leq 4 \\
	&\Longleftrightarrow &(x+3\leq 4 \wedge x+3 \geq 0) &\vee (-x-3\leq 4 \wedge x+3 \leq 0) \\
	&\Longleftrightarrow & (-3 \leq x \leq 1) &\vee (-7 \leq x \leq -3) \\
	&\Longleftrightarrow & -7 \leq &x \leq 1 \\
\end{alignat*}

Donc : $S=[-7 ; 1]$

\subsubsection*{(c)}

\begin{alignat*}{2}
	&                    & |x+1| &>3  \\
	&\Longleftrightarrow &(x+1 > 3 \wedge x+1 \geq 0) &\vee (-x-1 >3 \wedge x+1 \leq 0) \\
	&\Longleftrightarrow & (x > 2) &\vee (x < -4)
\end{alignat*}

Donc : $S=]-\infty ; -4] \cup ]2 ; +\infty[$

\subsubsection*{(d)}

\begin{tikzpicture}
	\tkzTabInit{$x$ / 1   , $|2x-4|$ / 1, $|x+2|$/1, Equation/1, S/1}
	           {$-\infty$ ,$-2$, $2$    , $+\infty$}
	\tkzTabLine{, -2x+4       , , -2x+4, , 2x-4 , }
	\tkzTabLine{, -x-2        , , x+2, , x+2 , }
	\tkzTabLine{, -x+6=0      , , -3x+2=0, , x-6=0 , }
	\tkzTabLine{, \varnothing ,\cup , \left\lbrace \frac{2}{3}\right\rbrace ,\cup , \left\lbrace 6 \right\rbrace , }
\end{tikzpicture}

Donc : $S=\left\lbrace \frac{2}{3} ; 6 \right\rbrace$

\subsubsection*{(e)}

\begin{tikzpicture}
	\tkzTabInit{$x$ / 1   , $|2x+4|$ / 1, $|x+1|$/1,Equation/1, S/1}
	{$-\infty$ ,$-2$, $-1$    , $+\infty$}
	\tkzTabLine{, -2x-4 , , 2x+4, , 2x+4 , }
	\tkzTabLine{, -x-1 , , -x-1, , x+1 , }
	\tkzTabLine{, -x-3 \leq 0 , , 3x+5 \leq 0, , x+3 \leq 0 , }
	\tkzTabLine{, [-3 ;-2] ,\cup , [-2 ; -\frac{5}{3}] ,\cup , \varnothing , }
\end{tikzpicture}

Donc : $S=\left[ -3 ; -\frac{5}{3} \right] $

\subsubsection*{(f)}

\begin{tikzpicture}
	\tkzTabInit{$x$ / 1   , $|x+1|$ / 1, $|x|$/1,Equation/1, S/1}
	{$-\infty$ ,$-1$, $0$    , $+\infty$}
	\tkzTabLine{, -x-1 , , x+1, , x+1 , }
	\tkzTabLine{, -x , , -x, , x , }
	\tkzTabLine{, 0=0 , , 2x+2=0, , 2=0 , }
	\tkzTabLine{, ]-\infty ;-1] ,\cup , \left\lbrace -1 \right\rbrace , \cup, \varnothing , }
\end{tikzpicture}

Donc : $S= ]-\infty ; -1]$

\subsection*{Exercice 2.5}

\begin{tikzpicture}
	\tkzTabInit{$x$ / 1   , $|2x-1|$ / 1, $|3x+1|$/1,Equation/1, S/1}
	           {$-\infty$ ,$-\frac{1}{3}$, $\frac{1}{2}$    , $+\infty$}
	\tkzTabLine{, -2x+1  , , -2x+1  , , 2x-1    , }
	\tkzTabLine{, -3x-1  , ,  3x+1  , , 3x+1    , }
	\tkzTabLine{, 4x+3=1 , , -8x-1=1, , -4x-3=1 , }
	\tkzTabLine{,   \left\lbrace -\frac{1}{2} \right\rbrace,\cup , \left\lbrace -\frac{1}{4} \right\rbrace , \cup, \varnothing , }
\end{tikzpicture}

Donc : $S=\left\lbrace -\frac{1}{2} ; -\frac{1}{4} \right\rbrace$

\subsection*{Exercice 2.6}

\begin{equation*}
	\begin{split}
		\forall x \in [-3 ; 1],
		& \begin{cases}
			0 \leq |x+1| \leq 2 \\
		    -1 \leq x+2 \leq 3 \\
		    1 \leq x^2 + 1 \leq 10
		\end{cases} \\
		\Longrightarrow 
		& \begin{cases}
			-2 \leq |x+1|(x+2) \leq 6 \\
			0 < \frac{1}{x^2 + 1} \leq 1
		\end{cases} \\
		\Longrightarrow 
		& -2 \leq \frac{|x+1|(x+2)}{x^2 + 1} \leq 6 
	\end{split}
\end{equation*}

\subsection*{Exercice 2.7}

\subsubsection*{1. Egalité de l'inégalité triangulaire}

L'égalité est évidemment vraie si $x$ et $y$ sont de même signe :
\begin{displaymath}
	xy \geq 0 \Longrightarrow |x+y| = |x| + |y|
\end{displaymath}

Plutôt que de prouver la réciproque ($|x+y| = |x| + |y| \Longrightarrow xy \geq 0$), démontrons
la contraposée : $xy < 0 \Longrightarrow |x+y| \neq |x| + |y|$.

Soient $x$ et $y$ tels que $ x < 0 < y$, on alors :

\begin{displaymath}
		|x+y| - |x| - |y|
		= \begin{cases}
			x+y-(-x)-y = 2x \neq 0 \text { si } y \geq -x \\
			-x-y-(-x)-y = -2y \neq 0 \text { si } y \leq -x \\
		\end{cases}
\end{displaymath}

L'égalité n'est donc pas vérifiée si les 2 variables n'ont pas le même signe ce qui démontre la contraposée
et donc l'équivalence.

\subsubsection*{2. min et max}

Les 2 fonctions étant symétriques en $x$ et $y$ (car $|x-y| = |y-x]$), il suffit de vérifier les égalités
pour $x \leq y$ et elles seront vraies pour $x \geq y$.

Supposons $x \leq y$ (et donc $|x-y|=y-x$), alors :
\begin{equation*}
	\begin{split}
		\frac{x + y + |x-y|}{2} &= \frac{x + y + (y-x)}{2} \\
		                        &= y \\
		                        &= max(x,y)
	\end{split}
\end{equation*}

et

\begin{equation*}
	\begin{split}
		\frac{x + y - |x-y|}{2} &= \frac{x + y - (y-x)}{2} \\
		&= x \\
		&= min(x,y)
	\end{split}
\end{equation*}

\subsection*{Exercice 2.8}

\subsubsection*{1.}
\begin{equation*}
	\begin{split}
		&\frac{\exp(x^2-1)}{\exp(x+1)} \leq 1 \\
		\Longleftrightarrow &\exp(x^2-1 -(x+1)) \leq 1 \\
		\Longleftrightarrow &\exp((x+1)(x-2)) \leq 1 \\
		\Longleftrightarrow &(x+1)(x-2) \leq 0 \\
		\Longleftrightarrow & -1 \leq x \leq 2 \\
	\end{split}
\end{equation*}

Donc : $S= [-1 ; 2]$

\subsubsection*{2.}
\begin{equation*}
	\begin{split}
		& e^{2x} + e^x -6 = 0 \\
		\Longleftrightarrow & e^x = \frac{-1\pm \sqrt{25}}{2} \\
		\Longrightarrow & e^x = 2 \text{ car $e^x$ est strictement positif} \\
		\Longrightarrow & x = \ln 2
	\end{split}
\end{equation*}

Donc : $S=\left\lbrace \ln 2 \right\rbrace$

\subsubsection*{3.}
\begin{equation*}
	\begin{split}
		\forall x \in \mathbb{R} \setminus \left\lbrace -1 ; 1 \right\rbrace, & \ln |x+1| - \ln |x-1| \leq \ln 2 \\
		\Longleftrightarrow & \ln \left| \frac{x+1}{x-1} \right| \leq \ln 2 \\
		\Longleftrightarrow & \left| \frac{x+1}{x-1} \right| \leq 2 \text{ car $\ln$ est croissante sur $\mathbb{R}^{+}_{*}$} \\
	\end{split}
\end{equation*}


Soit $f(x) = \left| \frac{x+1}{x-1} \right|$ définie sur $\mathbb{R} \setminus \left\lbrace -1 ; 1 \right\rbrace$.

\begin{displaymath}
	\lim_{x \rightarrow -\infty} f(x) = 1^{-}
\end{displaymath}

\begin{displaymath}
	\lim_{x \rightarrow -1} f(x) = 0
\end{displaymath}

\begin{displaymath}
	\lim_{x \rightarrow 1-} f(x) = +\infty
\end{displaymath}

\begin{displaymath}
	\lim_{x \rightarrow 1+} f(x) = +\infty
\end{displaymath}

\begin{displaymath}
	\lim_{x \rightarrow +\infty} f(x) = 1^{+}
\end{displaymath}

\begin{tikzpicture}
	\tkzTabInit{$x$ / 1   , $|x+1|$ / 1, $|x-1|$/1,$f(x)$/1, $f'(x)$/1, Variations/2}
	           {$-\infty$  ,$-1$ , $1$    , $+\infty$}
	\tkzTabLine{, -x-1  , , x+1   , , x+1    , }
	\tkzTabLine{, -x+1  , , -x+1  , , x-1    , }
	\tkzTabLine{, \frac{x+1}{x-1} , , -\frac{x+1}{x-1}, , \frac{x+1}{x-1} , }
	\tkzTabLine{, -\frac{2}{(x-1)^2}  , , \frac{2}{(x-1)^2} , , -\frac{2}{(x-1)^2} , }
	\tkzTabVar{+/$1^{-}$, -/$0$ , +/$+\infty$, -/$1^{+}$ , }
\end{tikzpicture}

L'équation $f(x)=2$ présente donc 1 racine sur l'intervalle $]-1 ; 1[$ et une autre sur $]1; +\infty[$ : $\frac{1}{3}$ et $3$

On a donc $S = ]-\infty ; \frac{1}{3}] \cup [3 ; +\infty[$
\end{document}]