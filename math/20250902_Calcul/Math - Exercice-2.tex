% XeLaTeX can use any Mac OS X font. See the setromanfont command below.
% Input to XeLaTeX is full Unicode, so Unicode characters can be typed directly into the source.

% The next lines tell TeXShop to typeset with xelatex, and to open and save the source with Unicode encoding.

%!TEX TS-program = xelatex
%!TEX encoding = UTF-8 Unicode

\documentclass{report}
\usepackage[french]{babel}
\usepackage{amsmath}
\usepackage{geometry}                % See geometry.pdf to learn the layout options. There are lots.
\geometry{a4paper,hmargin=2cm,vmargin=1.5cm,includeheadfoot}                   % ... or a4paper or a5paper or ... 
%\geometry{landscape}                % Activate for for rotated page geometry
%\usepackage[parfill]{parskip}    % Activate to begin paragraphs with an empty line rather than an indent
\usepackage{graphicx}
\usepackage{amssymb}
\usepackage{tkz-tab}
\usepackage{pgfplots}
\pgfplotsset{compat=1.18}


%\usepackage{fontspec,xltxtra,xunicode}
%\defaultfontfeatures{Mapping=tex-text}
%\setromanfont[Mapping=tex-text]{Hoefler Text}
%\setsansfont[Scale=MatchLowercase,Mapping=tex-text]{Gill Sans}
%\setmonofont[Scale=MatchLowercase]{Andale Mono}

\title{Math - Exercices}
\author{Poppy RAVEZ}
%\date{}                                           % Activate to display a given date or no date

\begin{document}


\chapter*{2 - Fonctions réelles usuelles}


\subsection*{Application 2.3}

\subsubsection*{a. $\forall x \in \mathbb{R}, x \leq |x|$}

\begin{displaymath}
	\begin{cases}
		\forall x \in \mathbb{R}^{+}, |x| = x \geq x \\
		\forall x \in \mathbb{R}^{-}_{*}, |x| = -x > 0 > x
	\end{cases}
\end{displaymath}
Donc $\forall x \in \mathbb{R}, x \leq |x|$

\subsubsection*{b. $\forall x \in \mathbb{R}, |x| = 0 \Longleftrightarrow x = 0$}

Par définition : $x = 0 \Longrightarrow |x| = 0$.

Si $|x| = 0$, les solutions  vérifient :

\begin{displaymath}
	\begin{cases}
		x = |x| = 0 \text{ si } x \in \mathbb{R}^{+}\\
		-x = |x| = 0 \text{ si } x \in \mathbb{R}^{-}_{*}
	\end{cases}
\end{displaymath}

Donc $x=0$

L'implication réciproque $|x|=0, \Longrightarrow x=0$ est donc vraie ce qui démontre l'équivalence.

\subsubsection*{c. $\forall (x,y) \in \mathbb{R}^2, |xy| = |x|.|y|$}

\begin{tabular}{|c|c|c|}
	\hline
	$x$                  & $y$                  & $|xy|$\\
	\hline
	$\mathbb{R}^{+}$     & $\mathbb{R}^{+}$     & $|xy| = xy = |x|.|y|$ \\
	\hline
	$\mathbb{R}^{+}$     & $\mathbb{R}^{-}_{*}$ & $|xy| = -xy = x(-y) = |x|.|y|$ \\
	\hline
	$\mathbb{R}^{-}_{*}$ & $\mathbb{R}^{+}$     & $|xy| = -xy = (-x)y = |x|.|y|$ \\
	\hline
	$\mathbb{R}^{-}_{*}$ & $\mathbb{R}^{-}_{*}$ & $|xy| = xy =(-x)(-y) = |x|.|y|$ \\
	\hline
\end{tabular}

CQFD

Dans l'absolu et pour coller à la définition, il faudrait distinguer les cas $xy<0$ et $xy=0$ quand les deux variables n'ont pas
le même signe.


\subsubsection*{d. $\forall x \in \mathbb{R}, \forall z \in \mathbb{R}^{+}, |x| \leq z \Longleftrightarrow -z \leq x \leq z$}

Soit $z \in \mathbb{R}^{+}$ (et donc $-z \leq 0$).

Si $x \in \mathbb{R}^{+}$ (et donc $|x|=x \geq 0$):

\begin{itemize}
	\item $|x| \leq z \Longrightarrow -z \leq 0 \leq x \leq z \Longrightarrow -z \leq x \leq z$
	\item $-z \leq x \leq z \Longrightarrow -z \leq 0 \leq x \leq z \Longrightarrow |x| \leq z$
\end{itemize}

Ce qui démontre l'équivalence sur $\mathbb{R}^{+}$ ; l'équivalence sur $\mathbb{R}^{-}$ en est la conséquence
en remplaçant $x$ par $-x$ (les deux inégalités étant invariantes par changement de signe).


\subsection*{Application 2.34 - Dérivabilité de $p_{\alpha}$ en $0$ avec $\alpha \geq 1$}

$p_{\alpha}$ est définie sur $\mathbb{R}^{+}_{*}$ par $p_{\alpha}(x) = \exp(\alpha \ln(x))$, cependant :
\begin{displaymath}
	\lim_{x\rightarrow 0^{+}} \ln(x) = -\infty \Longrightarrow\forall \alpha>0, \lim_{x\rightarrow 0^{+}} p_{\alpha}(x) = 0
\end{displaymath}

On peut donc, si $\alpha > 0 0$, définir par continuité la fonction $p_{\alpha}$ en $0$ avec la valeur nulle.

On a alors :
\begin{equation*}
	\begin{split}
		\lim_{h\rightarrow 0^{+}} \frac{p_{\alpha}(h) - 0}{h}
		&= \lim_{h\rightarrow 0^{+}} \frac{e^{\alpha \ln(h)}}{h} \\
		&= \lim_{h\rightarrow 0^{+}} \frac{e^{\alpha \ln(h)}}{e^{\ln(h)}} \\
		&= \lim_{h\rightarrow 0^{+}} e^{(\alpha-1) \ln(h)} \\
		&=0 \text{ si $\alpha >1$ car alors $\lim_{h\rightarrow 0^{+}} (\alpha-1) \ln(h) = -\infty$}
	\end{split}
\end{equation*}

La fonction $p_{\alpha}$ prolongée en $0$ avec $p_{\alpha}(0)=0$ est donc dérivable à droite en $0$ si $\alpha>1$
avec $p'_{\alpha}(0)=0$.


\subsection*{Application 2.41 - Limite en $0$ de $x^x$}

\begin{equation*}
	\begin{split}
		\lim_{x\rightarrow 0^{+}} x^x
		&= \lim_{x\rightarrow 0^{+}} e^{x\ln(x)} \\
	    &= 1 \text{ car  $\lim_{x\rightarrow 0^{+}} x\ln(x) = 0$ d'après la proposition 2.39}
	\end{split}
\end{equation*}


\subsection*{Exercice 2.1}

\begin{equation*}
	\begin{split}
		\forall x \in \mathbb{R}, \lim_{h\rightarrow 0} \dfrac{\dfrac{1}{1+(x+h)^2}-\dfrac{1}{1+x^2}}{h}
		&= \lim_{x\rightarrow 0} \dfrac{1+x^2-1-(x+h)^2}{h(1+(x+h)^2)(1+x^2)} \\
		&= \lim_{x\rightarrow 0} \dfrac{-2xh-h^2}{h(1+(x+h)^2)(1+x^2)} \\
		&= \lim_{x\rightarrow 0} \dfrac{-2x-h}{(1+(x+h)^2)(1+x^2)} \\
		&= \frac{-2x}{(1+x^2)^2} \\
	\end{split}
\end{equation*}

Car la limite du dénominateur est un réel strictement positif.

La fonction $f$ est donc dérivable sur $\mathbb{R}$ et $f'(x) = \frac{-2x}{(1+x^2)^2}$

On constate de manière évidente que $f$ est paire, strictement positive, atteint son maximum en $0$ et que :
\begin{displaymath}
	\lim_{x \rightarrow \pm \infty} f(x) = 0^{+}
\end{displaymath}


On en déduit le tableau de variation suivant :

\begin{tikzpicture}
	\tkzTabInit{$x$ / 1 , $f'(x)$ / 1, $f(x)$/2}{$-\infty$ ,$0$, $+\infty$}
	\tkzTabLine{, + , 0 , - , }
	\tkzTabVar{-/$0$, +/$1$, -/$0$}
\end{tikzpicture}

$f$ possède deux points d'inflexion (où elle traverse sa tangente) caractérisés par $f''(x)=0$ :

\begin{equation*}
	\begin{split}
		f''(x) &= \dfrac{-2(1+x^2)^2-2(-2x)(2x)(1+x^2)}{(1+x^2)^4} \\
		       &= \dfrac{-2(1+x^2)+8x^2}{(1+x^2)^3} \\
		       &= 2\dfrac{3x^2-1}{(1+x^2)^3} 
	\end{split}
\end{equation*}

Les points d'inflexion sont donc $\binom{\pm\frac{1}{\sqrt{3}}}{\frac{3}{4}}$.

\begin{tikzpicture}
	\begin{axis}[
		axis lines=center,
		height=5cm,
		width=10cm,
		xlabel={$x$},
		ylabel={$f(x)$},
		xmin=-4, xmax= 4,
		ymin=0, ymax=1.1,
		]
		\addplot[blue,samples=81]{1/(1+x^2)};
	\end{axis}	
\end{tikzpicture}

\end{document}  