% XeLaTeX can use any Mac OS X font. See the setromanfont command below.
% Input to XeLaTeX is full Unicode, so Unicode characters can be typed directly into the source.

% The next lines tell TeXShop to typeset with xelatex, and to open and save the source with Unicode encoding.

%!TEX TS-program = xelatex
%!TEX encoding = UTF-8 Unicode

\documentclass[a4paper,10pt]{report}
\usepackage[french]{babel}
\usepackage{standalone}
\usepackage{amsmath}
\usepackage{geometry}                % See geometry.pdf to learn the layout options. There are lots.
\geometry{a4paper,hmargin=2cm,vmargin=1.5cm,includeheadfoot}                   % ... or a4paper or a5paper or ... 
%\geometry{landscape}                % Activate for for rotated page geometry
%\usepackage[parfill]{parskip}    % Activate to begin paragraphs with an empty line rather than an indent
\usepackage{graphicx}
\usepackage{amssymb}
\usepackage{tkz-tab}
\usepackage{pgfplots}
\pgfplotsset{compat=1.18}
%\usepackage{xcolor}
%\definecolor{dauphineblue}{RGB}{47,68,134}


%\usepackage{fontspec,xltxtra,xunicode}
%\defaultfontfeatures{Mapping=tex-text}
%\setromanfont[Mapping=tex-text]{Hoefler Text}
%\setsansfont[Scale=MatchLowercase,Mapping=tex-text]{Gill Sans}
%\setmonofont[Scale=MatchLowercase]{Andale Mono}

\title{Math - Exercices}
\author{Poppy RAVEZ}
%\date{}                                           % Activate to display a given date or no date

\begin{document}
	
\subsection*{Application 4.6}

La fonction cosinus étant paire, on a : $\cos|x| = \cos x$ ; $\cos|x|$ est donc dérivable car $\cos$ l'est.

La dérivée à droite de $\sin |x|$ en $0$, est celle de $\sin x$ soit $\cos 0 = 1$.

La dérivée à gauche de $\sin |x|$ en $0$ est la dérivée à gauche de $\sin -x$ soit $-\cos 0 = -1$.

$\sin |x|$ n'est donc pas dérivable en $0$.

\subsection*{Application 4.8}

Calcul trivial en utilisant les formulesde $\cos(a+b)$ et $\sin(a+b)$ :
\begin{itemize}
	\item[•] $\cos(\pi-x) = -\cos x$
	\item[•] $\sin(x- \frac{\pi}{2}) = -\cos x$ =
\end{itemize}

\subsection*{Application 4.9}

\begin{displaymath}
	\cos\left( \frac{2\pi}{3}\right) = - \cos\left( \pi - \frac{2\pi}{3}\right) = -\cos\left( \frac{\pi}{3}\right) = -\frac{1}{2}
\end{displaymath}

\begin{displaymath}
	\sin\left( \frac{2\pi}{3}\right) =  \sin\left( \pi - \frac{2\pi}{3}\right) = \sin\left( \frac{\pi}{3}\right) = \frac{\sqrt{3}}{2}
\end{displaymath}

\subsection*{Application 4.14}

\begin{equation*}
	\begin{split}
		\tan(a) + \tan b &= \frac{\sin(a)}{\cos(a)} + \frac{\sin(b)}{\cos(b)} \\
		                 &=\frac{\sin(a)\cos(b) + \cos(a)\sin(b)}{\cos(a)\cos(b)} \\
		                 &= \frac{\sin(a+b)}{\cos(a)\cos(b)}
	\end{split}
\end{equation*}


\subsection*{Application 4.15}

D'après l'expression de $\tan(a+b)$, on a :
\begin{displaymath}
	\tan\left( 2\frac{\pi}{8} \right) = \tan\left( \frac{\pi}{4} \right)= 1 =\frac{2\tan\left( \frac{\pi}{8} \right)}{1-\tan^2\left( \frac{\pi}{8} \right)}
\end{displaymath}

Et donc :

\begin{displaymath}
	\tan^2\left( \frac{\pi}{8} \right) + 2\tan\left( \frac{\pi}{8} \right) -1 = 0
\end{displaymath}

$\tan\left( \frac{\pi}{8} \right)$ est donc la racine positive de l'équation $x^2+ 2x -1 = 0$, soit : $-1 + \sqrt{2}$

\subsection*{Application 4.17}

$\arcsin\left( -\frac{\sqrt{3}}{2}\right) = -\frac{\pi}{3} $

\subsection*{Application 4.19}

$\arccos\left( -\frac{1}{2}\right) = \frac{2\pi}{3} $

\subsection*{Application 4.21}

$\arctan\left( -\frac{1}{\sqrt{3}}\right) = -\frac{\pi}{6} $

\subsection*{Exercice 4.1}

On a :
\begin{displaymath}
	\cos\left( 2\frac{\pi}{8} \right) = \cos\left( \frac{\pi}{4} \right) = \frac{\sqrt{2}}{2} = 2\cos^2\left(\frac{\pi}{8}\right) -1
\end{displaymath}

$\cos\left(\frac{\pi}{8}\right)$ étant positif on a :
\begin{displaymath}
	\cos\left(\frac{\pi}{8}\right) = \frac{\sqrt{2+\sqrt{2}}}{2}
\end{displaymath}

$\sin\left(\frac{\pi}{8}\right)$ étant positif on a :
\begin{displaymath}
	\sin\left(\frac{\pi}{8}\right) = \sqrt{1-\cos^2\left(\frac{\pi}{8}\right)} = \frac{\sqrt{2-\sqrt{2}}}{2}
\end{displaymath}


\subsection*{Exercice 4.2}
\begin{equation*}
	\begin{split}
		\sin^2(a+b) + \cos^2(a+b) &= (\sin(a) \cos(b) + \cos(a)\sin(b))^2 + (\cos(a)\cos(b) - \sin(a)\sin(b))^2 \\
			&= \sin^2(a)\cos^2(b) + \cos^2(a)\sin^2(b) + 2\sin(a)\cos(b)\cos(a)\sin(b) \\
			&\phantom{=}+ \cos^2(a)\cos^2(b) + \sin^2(a)\sin^2(b) + 2\cos(a)\cos(b)\sin(a)\sin(b) \\
			&= \sin^2(a) (\cos^2(b) + \sin^2(b)) + \cos^2(a) (\sin^2(b) + \cos^2(b)) \\
			&\phantom{=}+ 4 \sin(a)\cos(a) \times \sin(b)\cos(b) \\
			&= \cos^2(a) + \sin^2(a) + \sin(2a)\sin(2b) \\
			&= 1 + \sin(2a)\sin(2b)
	\end{split}
\end{equation*}

\subsection*{Exercice 4.3}

\subsubsection*{4.3.1.a}
\begin{displaymath}
	\lim_{x \rightarrow 0} \frac{\cos x}{x^2} = +\infty
\end{displaymath}

\subsubsection*{4.3.1.b}
\begin{displaymath}
	\lim_{x \rightarrow 0} \frac{\tan^4\left( x+\frac{\pi}{2} \right) + \tan\left( x+\frac{\pi}{2} \right) - 1}{\tan^2\left( x+\frac{\pi}{2} \right) + 1} = +\infty
\end{displaymath}

\subsubsection*{4.3.1.c}
\begin{displaymath}
	\lim_{x \rightarrow +\infty} e^{\cos(\sin(\frac{1}{x}))} = e
\end{displaymath}

\subsubsection*{4.3.2.a}

La dérivée de $\cos(x)e^x$ est : $(\cos(x)-sin(x))e^x$

\subsubsection*{4.3.2.b}

La dérivée de $x^2\sin(x)$ est : $2x\sin(x) + x^2\cos(x)$

\subsubsection*{4.3.2.c}

La dérivée de $\tan(x)e^{-x}$ est : $(\tan^2(x)-\tan(x)+1)e^{-x}$

\subsubsection*{4.3.2.d}

La dérivée de $\frac{\cos(x)}{x^2-2x+2}$ est : $\frac{-(x^2-2x+2)\sin(x) - (2x-2)\cos(x)}{(x^2-2x+2)^2}$

\subsubsection*{4.3.2.e}

La dérivée de $e^{\cos(x)}$ est : $-\sin(x) e^{\cos(x)}$

\subsubsection*{4.3.2.f}

La dérivée de $\ln(\cos(x))$ est : $\frac{-\sin(x)}{\cos(x)} = -\tan x$

\subsubsection*{4.3.3.a}
\begin{displaymath}
	\int_0^\pi \sin(x)\cos(x)dx = \left[ \frac{1}{2}\sin^2 x \right]_0^\pi = 0
\end{displaymath}

\subsubsection*{4.3.3.b}
La fonction étant impaire et le domaine d'intégration symétrique par rapport à $0$, l'intégrale
est nulle mais on peut néanmoins la calculer :
\begin{displaymath}
	\int_{-1}^1 x\cos(x^2)dx = \left[ \frac{1}{2}\sin x^2 \right]_{-1}^1 = 0
\end{displaymath}

\subsubsection*{4.3.3.c}
La fonction étant impaire et le domaine d'intégration symétrique par rapport à $0$, l'intégrale
est nulle mais on peut néanmoins la calculer :
\begin{displaymath}
	\int_{-1}^1 x^2 \frac{\sin x^3}{\cos x^3} dx = \left[ -\frac{1}{3}\ln(\cos x^3) \right]_{-1}^1= 0
\end{displaymath}

\subsubsection*{4.3.3.d}
On a :
\begin{displaymath}
	\int_{-1}^1 \cos(x)e^x dx = \left[ \cos(x)e^x \right]_{-1}^1 - \int_{-1}^1 -\sin(x)e^x dx
\end{displaymath}
Mais aussi :
\begin{displaymath}
	\int_{-1}^1 \cos(x)e^x dx = \left[ \sin(x)e^x \right]_{-1}^1 - \int_{-1}^1 \sin(x)e^x dx
\end{displaymath}

En sommant les deux égalités, on obtient :
\begin{equation*}
	\begin{split}
	\int_{-1}^1 \cos(x)e^x dx &= \frac{1}{2}\left[ \cos(x)e^x \right]_{-1}^1 +  \frac{1}{2}\left[ \sin(x)e^x \right]_{-1}^1 \\
                              &= \frac{1}{2} (\cos(1)e - \cos(1)e^{-1} + \sin(1) e + \sin(1)e^{-1}) \\
                              &= \frac{e^2-1}{2e}\cos(1) + \frac{e^2+1}{2e} \sin(1)
	\end{split}
\end{equation*}

\subsubsection*{4.4}

Une primitive de $\tan x = \frac{\sin x}{\cos x}$ sur $]-\frac{\pi}{2} ; \frac{\pi}{2}[$ est $P(x)=-\ln(\cos(x))$

\end{document}