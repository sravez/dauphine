% XeLaTeX can use any Mac OS X font. See the setromanfont command below.
% Input to XeLaTeX is full Unicode, so Unicode characters can be typed directly into the source.

% The next lines tell TeXShop to typeset with xelatex, and to open and save the source with Unicode encoding.

%!TEX TS-program = xelatex
%!TEX encoding = UTF-8 Unicode

\documentclass[a4paper,10pt]{report}
\usepackage[french]{babel}
\usepackage{standalone}
\usepackage{amsmath}
\usepackage{geometry}                % See geometry.pdf to learn the layout options. There are lots.
\geometry{a4paper,hmargin=2cm,vmargin=1.5cm,includeheadfoot}                   % ... or a4paper or a5paper or ... 
%\geometry{landscape}                % Activate for for rotated page geometry
%\usepackage[parfill]{parskip}    % Activate to begin paragraphs with an empty line rather than an indent
\usepackage{graphicx}
\usepackage{amssymb}
\usepackage{tkz-tab}
\usepackage{pgfplots}
\pgfplotsset{compat=1.18}
\usepackage{algorithm,algorithmic}

%\usepackage{xcolor}
%\definecolor{dauphineblue}{RGB}{47,68,134}


%\usepackage{fontspec,xltxtra,xunicode}
%\defaultfontfeatures{Mapping=tex-text}
%\setromanfont[Mapping=tex-text]{Hoefler Text}
%\setsansfont[Scale=MatchLowercase,Mapping=tex-text]{Gill Sans}
%\setmonofont[Scale=MatchLowercase]{Andale Mono}

\title{Logique - Exercices}
\author{Poppy RAVEZ}
%\date{}                                           % Activate to display a given date or no date

\begin{document}
	
	
\subsection*{Exercice 1.11 - Système de deux équations linéaires}

Une équation linéaire à 2 inconnues de la forme $ax +by = c$ correspond dans le plan à :
\begin{itemize}
	\item une droite si $|a| + |b| > 0$
	\item le plan si $|a|+|b|= c = 0$
	\item l'ensemble vide sinon ($|a|+|b|=0$ et $c\neq0$)
\end{itemize}

Le résultat d'un système de 2 equations linéaires à 2 inconnues peut-donc être :
\begin{itemize}
	\item un point si on a deux droites non parallèles (donc sécantes) ;
	\item l'ensemble vide si une des équations n'a pas de solutions ou si les équations
	      correspondent à deux droites parallèles non confondues ;
	\item le plan si les deux équations sont triviales et correspondent au plan ;
	\item une droite si l'on a deux droites confondues ou un plan et une droite.
\end{itemize}

Soit le systèmes d'équations :

\begin{displaymath}
\begin{cases}
	ax + by = c \\
	dx + ey = f
\end{cases}
\end{displaymath}

Le cas des droites concourantes est caractérisé par $\Delta = ae - db \neq 0$ ; la solution est alors :
\begin{displaymath}
\begin{pmatrix}
	x \\ y
\end{pmatrix}
=	
\begin{pmatrix}
	\frac{\Delta_x}{\Delta} \\
	\frac{\Delta_x}{\Delta}
\end{pmatrix}
=
\begin{pmatrix}
	\frac{ce-fb}{ae-db} \\
	\frac{af-dc}{ae-db}
\end{pmatrix}
\end{displaymath}

Le cas $\Delta = 0$ ne peut être résumé à la nullité de $\Delta_x$ et $\Delta_y$ car les systèmes suivants
ont leurs 3 déterminants nuls mais ne possède aucune solutions :

\begin{displaymath}
	\begin{cases}
		0x + 0y = a & \text{avec } a \neq 0 \\
		0x + 0y = b & \text{avec } b \in \mathbb{R}
	\end{cases}
\end{displaymath}

Si on considère (toujour avec $\Delta = 0$ ) uniquement les systèmes où les équations linéaires ont des solutions 
(i.e. pas celle de la forme $0=a$ avec $a\neq 0$), le système n'a une solution que si $\Delta_x = \Delta_y = 0$ 


\begin{algorithm}
	\caption{Système d'équations linéaires}
	\begin{algorithmic}
		\STATE \textbf{Algo}
			\STATE $a$: \textbf{float}
			\STATE $b$: \textbf{float}
			\STATE $c$: \textbf{float}
			\STATE $d$: \textbf{float}
			\STATE $e$: \textbf{float}
			\STATE $f$: \textbf{float}
			\STATE $D$: \textbf{float}
			\STATE $DX$: \textbf{float}
			\STATE $DY$: \textbf{float}
			
		\STATE \textbf{begin}
		\PRINT "Première équation : ax+by = c"
		\PRINT "a ?"
		\STATE \textbf{get} $a$
		\PRINT "b ?"
		\STATE \textbf{get} $b$
		\PRINT "c ?"
		\STATE \textbf{get} $c$
		\PRINT "Seconde équation : dx+ey = f"
		\PRINT "d ?"
		\STATE \textbf{get} $d$
		\PRINT "e ?"
		\STATE \textbf{get} $e$
		\PRINT "f ?"
		\STATE \textbf{get} $f$
		
		\STATE $D \leftarrow ae-db$ 
		\STATE $DX \leftarrow ce-fb$
		\STATE $DY \leftarrow af-dc$
		
		\IF{$D \neq 0$}
			\PRINT "Solution unique"
			\PRINT "X = " + ($DX / D$)
			\PRINT "Y = " + ($DY / D$)
		\ELSIF{$(a=0 \wedge b=0 \wedge c\neq 0) \vee (d=0 \wedge e=0 \wedge f\neq 0) \vee (DX \neq 0) \vee (DY \neq 0)$}
			\PRINT "Aucune solution"
		\ELSIF{$(a=0) \wedge (b=0)$}
			\IF{$(d=0) \wedge (e=0)$}
				\PRINT "Solution = R x R"
			\ELSE
				\PRINT "Solution = droite d'équation " $dx + ey =f$
			\ENDIF
		\ELSE
			\PRINT "Solution = droite d'équation " $ax + by =c$
		\ENDIF
		\STATE \textbf{end}
	\end{algorithmic}
\end{algorithm}

\end{document}