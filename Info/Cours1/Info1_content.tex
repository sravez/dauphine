% XeLaTeX can use any Mac OS X font. See the setromanfont command below.
% Input to XeLaTeX is full Unicode, so Unicode characters can be typed directly into the source.

% The next lines tell TeXShop to typeset with xelatex, and to open and save the source with Unicode encoding.

%!TEX TS-program = xelatex
%!TEX encoding = UTF-8 Unicode

\documentclass[a4paper,10pt]{report}
\usepackage[french]{babel}
\usepackage{standalone}
\usepackage{amsmath}
\usepackage{geometry}                % See geometry.pdf to learn the layout options. There are lots.
\geometry{a4paper,hmargin=2cm,vmargin=1.5cm,includeheadfoot}                   % ... or a4paper or a5paper or ... 
%\geometry{landscape}                % Activate for for rotated page geometry
%\usepackage[parfill]{parskip}    % Activate to begin paragraphs with an empty line rather than an indent
\usepackage{graphicx}
\usepackage{amssymb}
\usepackage{tkz-tab}
\usepackage{pgfplots}
\pgfplotsset{compat=1.18}
\usepackage{algorithm}
%\usepackage{algorithmic}
\usepackage{algpseudocode}
\newcommand\Print{\State \textbf{print }}
\newcommand\Printtt[1]{\State \textbf{print }\texttt{"#1"}}
\newcommand\Get{\State \textbf{get }}
\algblock[<block>]{Vars}{EndVars}
\algblock[<block>]{Begin}{End}

\usepackage{diagbox}

%\usepackage{xcolor}
%\definecolor{dauphineblue}{RGB}{47,68,134}


%\usepackage{fontspec,xltxtra,xunicode}
%\defaultfontfeatures{Mapping=tex-text}
%\setromanfont[Mapping=tex-text]{Hoefler Text}
%\setsansfont[Scale=MatchLowercase,Mapping=tex-text]{Gill Sans}
%\setmonofont[Scale=MatchLowercase]{Andale Mono}

\title{Logique - Exercices}
\author{Poppy RAVEZ}
%\date{}                                           % Activate to display a given date or no date

\begin{document}


\subsection*{Exercice 1.1}
\begin{tabular}{|l|l|}
	\hline
	Instruction & Action \\
	\hline
	\texttt{x = 3.5}             & Affectation de la valeur $3.5$ à la variable \texttt{x}  \\
	\hline
	\textbf{print}\texttt{(x)}   & Affichage de la valeur de la variable \texttt{x}, donc \texttt{3.5} \\
	\hline
	\textbf{print}\texttt{("x")} & Affichage de la chaîne de caractère(s) \texttt{x} \\
	\hline
	\texttt{v}=\textbf{float}(\textbf{Input}(\texttt{Saisir...})) & Affectation à la variable \texttt{v} d'une entrée utilisateur convertie en flottant\\
	\hline
	\textbf{print}\texttt{("x=", v)}&  Affichage de la valeur précédente précédée de la chaîne \texttt{x=} \\
	\hline
\end{tabular}

La variable \texttt{x} n'est jamais modifiée après son initialisation et vaut donc $3.5$ à la fin de l'exécution.


\subsection*{Exercice 1.6}
\subsubsection*{1. $2.3$ et $-2$}
\texttt{moins de 4}

\texttt{a= 2.3}

\texttt{b= 1}

\subsubsection*{2. $4$ et $3$}
\texttt{plus de 4}

\texttt{a= 4.0}

\texttt{b= 0}

\subsubsection*{3. $-1$ et $1$}

\texttt{a= -2.0}

\texttt{b= 1}

\subsection*{Exercice 1.7}
\begin{algorithm}
	\caption{Système d'équations linéaires}
	\begin{algorithmic}
		\Vars
		\State $a$: \textbf{float}
		\State $b$: \textbf{float}
		\State $c$: \textbf{float}
		\EndVars
		
		\Statex
		\Begin
			\Printtt{a ?}
			\Get $a$
			\Printtt{b ?}
			\Get $b$
			\Printtt{c ?}
			\Get $c$
			
			\Statex
			\State $D \leftarrow b^2 - 4ac$ 
			\If{$D<0$}
				\Printtt{Aucune solution}
			\ElsIf{$D=0$}
				\Printtt{1 racine double : } + ($-b/2a$)
			\Else
				\Printtt{2 racines}
				\Printtt{r1 : } + ($(-b-\sqrt{D})/2a$)
				\Printtt{r2 : } + ($(-b+\sqrt{D})/2a$)
			\EndIf
		\End
	\end{algorithmic}
\end{algorithm}
	
\subsection*{Exercice 1.8}

On considère les paramètres comme entiers non nuls

\begin{tabular}{|l|c c c c c|}
	\hline
	message                                           & (niveau           & ,        &  section         & ,        & discipline)  $\in$\\
	\hline
	\texttt{il est trop petit}                        & $\{7,8,\ldots\}$ & $\times$ & $\mathbb{N}^*$   & $\times$ & $\mathbb{N}^*$ \\
	\hline
	\texttt{c'est un collégien}                       & $\{3, 4, 5, 6\}$ & $\times$ & $\mathbb{N}^*$   & $\times$ & $\mathbb{N}^*$ \\
	\hline
	\texttt{Lycéen qui n'est pas en section générale} & $\{1, 2\}$       & $\times$ & $\{1\}$          & $\times$ & $\mathbb{N}^*$\\
	\hline
	\texttt{Lycéen en section générale littéraire}    & $\{1, 2\}$       & $\times$ & $\{2,3,\ldots\}$ & $\times$ &  $\{1\}$\\
	\hline
	\texttt{Lycéen en section générale non littéraire}& $\{1, 2\}$       & $\times$ & $\{2,3,\ldots\}$ & $\times$ &  $\{2,3,\ldots\}$\\
	\hline
\end{tabular}
	

\subsection*{Exercice 1.9}
\begin{algorithm}
	\caption{Système d'équations linéaires}
	\begin{algorithmic}
		\Vars
			\State $n$: \textbf{int}
		\EndVars
		
		\Statex
		\Begin
			\Printtt{n ?}
			\Get $n$
			
			\Statex
			\If{$n<10$}
				\Printtt{Ajourné}
			\ElsIf{$n<12$}
				\Printtt{Passable}
			\ElsIf{$n<14$}
				\Printtt{AB}
			\ElsIf{$n<16$}
				\Printtt{B}
			\Else	
				\Printtt{TB}
			\EndIf
		\End
\end{algorithmic}
\end{algorithm}
	
\subsection*{Exercice 1.10}
\begin{algorithm}
	\caption{Système d'équations linéaires}
	\begin{algorithmic}
		\Vars
			\State $a$: \textbf{float}
			\State $b$: \textbf{float}
			\State $c$: \textbf{float}
		\EndVars
		
		\Statex
		\Begin
			\Printtt{a ?}
			\Get $a$
			\Printtt{b ?}
			\Get $b$
			\Printtt{c ?}
			\Get $c$
			
			\Statex
			\If{($(a=b+c)$ OR $(b=a+c)$ OR $(c=a+b)$)}
				\Printtt{oui}
			\Else
				\Printtt{non}
			\EndIf
		\End
	\end{algorithmic}
\end{algorithm}
	
\subsection*{Exercice 1.11 - Système de deux équations linéaires}

Une équation linéaire à 2 inconnues de la forme $ax +by = c$ correspond dans le plan à :
\begin{itemize}
	\item $D$ : une droite si $|a| + |b| > 0$
	\item $P$ = $\mathbb{R}^2$: le plan si $a=b=c=0$
	\item $\emptyset$ : l'ensemble vide sinon ($a=b=0$ et $c\neq0$)
\end{itemize}

Le résultat d'un système de 2 equations linéaires à 2 inconnues est l'intersection
de 2 des types d'éléments précédents et peut donc être :
\begin{itemize}
	\item un point si on a deux droites non parallèles (donc sécantes) ;
	\item l'ensemble vide si une des équations n'a pas de solutions ou si les équations
	      correspondent à deux droites parallèles non confondues ;
	\item le plan si les deux équations sont triviales et correspondent au plan ;
	\item une droite si l'on a deux droites confondues ou un plan et une droite.
\end{itemize}

Cela peut être résumé par le tableau suivant $S = S_1 \cap S_2$ :

\begin{tabular}{|c|c|c|p{5cm}|}
	\hline
	 \diagbox{$S_1$}{$S_2$}& $\emptyset$ & $P$         & $D_2$                            \tabularnewline
	\hline
	$\emptyset$            & $\emptyset$ & $\emptyset$ & $\emptyset$                      \tabularnewline
	\hline
	$P$                    & $\emptyset$ & $P = S_1$         & $D_2$                            \tabularnewline
	\hline
	$D_1$                  & $\emptyset$ & $D_1$       & Un point si droites concourantes \newline
	                                                     $D_2$ ($=D_1$) si confondues     \newline
	                                                     $\emptyset$ sinon                \tabularnewline
	\hline
\end{tabular}

Soit le systèmes d'équations :

\begin{displaymath}
\begin{cases}
	ax + by = c \\
	dx + ey = f
\end{cases}
\end{displaymath}

Le cas des droites concourantes est caractérisé par $\Delta = ae - db \neq 0$ : les vecteurs normaux aux droites, 
$\begin{pmatrix} a \\ b \end{pmatrix}$ et $\begin{pmatrix} d \\ e \end{pmatrix}$, sont non colinéaires.
La solution est alors :

\begin{displaymath}
\begin{pmatrix}
	x \\ y
\end{pmatrix}
=	
\begin{pmatrix}
	\frac{\Delta_x}{\Delta} \\
	\frac{\Delta_x}{\Delta}
\end{pmatrix}
=
\begin{pmatrix}
	\frac{ce-fb}{ae-db} \\
	\frac{af-dc}{ae-db}
\end{pmatrix}
\end{displaymath}

Deux droites (donc $|a|+|b| >0$ et $|d|+|e| >0$) sont confondues si les deux équations sont proportionnelles ce qui est équivalent à dire que
$\Delta=\Delta_x = \Delta_y = 0$.

Si on considère (toujours avec $\Delta = 0$) uniquement les systèmes où les équations linéaires ont des solutions 
(i.e. pas celle de la forme $0=a$ avec $a\neq 0$), le système a une solution ssi $\Delta_x = \Delta_y = 0$ :

\begin{tabular}{|c|c|c|c|}
	\hline
	\diagbox{$S_1$}{$S_2$}& $\emptyset$                             & $P$                           & $D_2$                            \\
	\hline
	$\emptyset$           & $\Delta=\Delta_x=\Delta_y=0$           & $\Delta=\Delta_x=\Delta_y=0$  & $\Delta=0$ ; $|\Delta_x|+|\Delta_y| >0$ \\
	\hline
	$P$                   & $\Delta=\Delta_x=\Delta_y=0$           & $\Delta=\Delta_x=\Delta_y=0$  & $\Delta=\Delta_x=\Delta_y=0$   \\
	\hline
	$D_1$                 & $\Delta=0$ ; $|\Delta_x|+|\Delta_y| >0$ & $\Delta=\Delta_x=\Delta_y=0$ &                                \\
	\hline
\end{tabular}
 


\begin{algorithm}
	\caption{Système d'équations linéaires}
	\begin{algorithmic}
		\Vars
			\State $x_1$: \textbf{float}
			\State $y_1$: \textbf{float}
			\State $c_1$: \textbf{float}
			\State $x_2$: \textbf{float}
			\State $y_2$: \textbf{float}
			\State $c_2$: \textbf{float}
			\State $D$: \textbf{float}
			\State $DX$: \textbf{float}
			\State $DY$: \textbf{float}
		\EndVars
		
		\Statex
		\Begin
			\Printtt{Première équation : aX+bY = c} \Comment{Saisie des variables}
			\Printtt{a ?}
			\Get $x_1$
			\Printtt{b ?}
			\Get $y_1$
			\Printtt{c ?}
			\Get $c_1$
			
			\Printtt{Seconde équation : dX+eY = f}
			\Printtt{d ?}
			\Get $x_2$
			\Printtt{e ?}
			\Get $y_2$
			\Printtt{f ?}
			\Get $c_2$
			
			\Statex \Comment{Calcul des déterminants}
			\State $D \leftarrow x_1 y_2-x_2y_1$ 
			\State $D_X \leftarrow c_1y_2-c_2y_1$
			\State $D_Y \leftarrow x_1c_2-x_2c_1$
			
			\Statex
			\If{$D \neq 0$}
				\Printtt{Solution unique}
				\Printtt{X = } + ($D_X / D$)
				\Printtt{Y = } + ($D_Y / D$)
			\ElsIf{$(x_1=0 \wedge y_1=0 \wedge c_1\neq 0) \vee (x_2=0 \wedge y_2=0 \wedge c_2\neq 0) \vee (D_X \neq 0) \vee (D_Y \neq 0)$}
				\Printtt{Aucune solution}
			\ElsIf{$(x_1=0) \wedge (y_1=0)$}
				\If{$(x_2=0) \wedge (y_2=0)$}
					\Printtt{Solution = R x R}
				\Else
					\Printtt{Solution = droite d'équation } $x_2X + y_2Y =c_2$
				\EndIf
			\Else
				\Printtt{Solution = droite d'équation } $x_1X + y_1Y =c_1$
			\EndIf
		\End
	\end{algorithmic}
\end{algorithm}

\end{document}