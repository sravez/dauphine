% XeLaTeX can use any Mac OS X font. See the setromanfont command below.
% Input to XeLaTeX is full Unicode, so Unicode characters can be typed directly into the source.

% The next lines tell TeXShop to typeset with xelatex, and to open and save the source with Unicode encoding.

%!TEX TS-program = xelatex
%!TEX encoding = UTF-8 Unicode

\documentclass[a4paper,10pt]{report}
\usepackage[french]{babel}
\usepackage{standalone}
\usepackage{amsmath}
\usepackage{geometry}                % See geometry.pdf to learn the layout options. There are lots.
\geometry{a4paper,hmargin=2cm,vmargin=1.5cm,includeheadfoot}                   % ... or a4paper or a5paper or ... 
%\geometry{landscape}                % Activate for for rotated page geometry
%\usepackage[parfill]{parskip}    % Activate to begin paragraphs with an empty line rather than an indent
\usepackage{graphicx}
\usepackage{amssymb}
\usepackage{tkz-tab}
\usepackage{pgfplots}
\pgfplotsset{compat=1.18}
\usepackage{algorithm}
%\usepackage{algorithmic}
\usepackage{algpseudocode}
\newcommand\Print{\State \textbf{print }}
\newcommand\Printtt[1]{\State \textbf{print }\texttt{"#1"}}
\newcommand\Get{\State \textbf{get }}
\algblock[<block>]{Vars}{EndVars}
\algblock[<block>]{Begin}{End}

%\usepackage{xcolor}
%\definecolor{dauphineblue}{RGB}{47,68,134}


%\usepackage{fontspec,xltxtra,xunicode}
%\defaultfontfeatures{Mapping=tex-text}
%\setromanfont[Mapping=tex-text]{Hoefler Text}
%\setsansfont[Scale=MatchLowercase,Mapping=tex-text]{Gill Sans}
%\setmonofont[Scale=MatchLowercase]{Andale Mono}

\title{Logique - Exercices}
\author{Poppy RAVEZ}
%\date{}                                           % Activate to display a given date or no date

\begin{document}
	
	
\subsection*{Exercice 1.11 - Système de deux équations linéaires}

Une équation linéaire à 2 inconnues de la forme $ax +by = c$ correspond dans le plan à :
\begin{itemize}
	\item une droite si $|a| + |b| > 0$
	\item le plan si $|a|+|b|= c = 0$
	\item l'ensemble vide sinon ($|a|+|b|=0$ et $c\neq0$)
\end{itemize}

Le résultat d'un système de 2 equations linéaires à 2 inconnues peut-donc être :
\begin{itemize}
	\item un point si on a deux droites non parallèles (donc sécantes) ;
	\item l'ensemble vide si une des équations n'a pas de solutions ou si les équations
	      correspondent à deux droites parallèles non confondues ;
	\item le plan si les deux équations sont triviales et correspondent au plan ;
	\item une droite si l'on a deux droites confondues ou un plan et une droite.
\end{itemize}

Soit le systèmes d'équations :

\begin{displaymath}
\begin{cases}
	ax + by = c \\
	dx + ey = f
\end{cases}
\end{displaymath}

Le cas des droites concourantes est caractérisé par $\Delta = ae - db \neq 0$ ; la solution est alors :
\begin{displaymath}
\begin{pmatrix}
	x \\ y
\end{pmatrix}
=	
\begin{pmatrix}
	\frac{\Delta_x}{\Delta} \\
	\frac{\Delta_x}{\Delta}
\end{pmatrix}
=
\begin{pmatrix}
	\frac{ce-fb}{ae-db} \\
	\frac{af-dc}{ae-db}
\end{pmatrix}
\end{displaymath}

Le cas $\Delta = 0$ ne peut être résumé à la nullité de $\Delta_x$ et $\Delta_y$ car les systèmes suivants
ont leurs 3 déterminants nuls mais ne possède aucune solutions :

\begin{displaymath}
	\begin{cases}
		0x + 0y = a & \text{avec } a \neq 0 \\
		0x + 0y = b & \text{avec } b \in \mathbb{R}
	\end{cases}
\end{displaymath}

Si on considère (toujour avec $\Delta = 0$ ) uniquement les systèmes où les équations linéaires ont des solutions 
(i.e. pas celle de la forme $0=a$ avec $a\neq 0$), le système n'a une solution que si $\Delta_x = \Delta_y = 0$ 


\begin{algorithm}
	\caption{Système d'équations linéaires}
	\begin{algorithmic}
		\Vars
			\State $x_1$: \textbf{float}
			\State $y_1$: \textbf{float}
			\State $c_1$: \textbf{float}
			\State $x_2$: \textbf{float}
			\State $y_2$: \textbf{float}
			\State $c_2$: \textbf{float}
			\State $D$: \textbf{float}
			\State $DX$: \textbf{float}
			\State $DY$: \textbf{float}
		\EndVars
		
		\Statex
		\Begin
			\Printtt{Première équation : aX+bY = c} \Comment{Saisie des variables}
			\Printtt{a ?}
			\Get $x_1$
			\Printtt{b ?}
			\Get $y_1$
			\Printtt{c ?}
			\Get $c_1$
			
			\Printtt{Seconde équation : dX+eY = f}
			\Printtt{d ?}
			\Get $x_2$
			\Printtt{e ?}
			\Get $y_2$
			\Printtt{f ?}
			\Get $c_2$
			
			\Statex \Comment{Calcul des déterminants}
			\State $D \leftarrow x_1 y_2-x_2y_1$ 
			\State $D_X \leftarrow c_1y_2-c_2y_1$
			\State $D_Y \leftarrow x_1c_2-x_2c_1$
			
			\Statex
			\If{$D \neq 0$}
				\Printtt{Solution unique}
				\Printtt{X = } + ($D_X / D$)
				\Printtt{Y = } + ($D_Y / D$)
			\ElsIf{$(x_1=0 \wedge y_1=0 \wedge c_1\neq 0) \vee (x_2=0 \wedge y_2=0 \wedge c_2\neq 0) \vee (D_X \neq 0) \vee (D_Y \neq 0)$}
				\Printtt{Aucune solution}
			\ElsIf{$(x_1=0) \wedge (y_1=0)$}
				\If{$(x_2=0) \wedge (y_2=0)$}
					\Printtt{Solution = R x R}
				\Else
					\Printtt{Solution = droite d'équation } $x_2X + y_2Y =c_2$
				\EndIf
			\Else
				\Printtt{Solution = droite d'équation } $x_1X + y_1Y =c_1$
			\EndIf
		\End
	\end{algorithmic}
\end{algorithm}

\end{document}